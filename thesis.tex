\documentclass[12pt]{report}

\usepackage{multirow}
\usepackage{longtable}
\usepackage[table]{xcolor}
\usepackage[latin2]{inputenc}
\usepackage{amsmath}
\usepackage[pdftex]{graphicx}
\usepackage{amsfonts}
\usepackage{amssymb}
%\usepackage{fancyhdr}
\usepackage[left=1.5in]{geometry}
\usepackage{geometry}
%\pagestyle{fancyplain}
\usepackage{wrapfig}
\usepackage{setspace}
\usepackage{epsfig}
\usepackage{color}
\usepackage{tikz}
\usepackage{build-aux/UVMThesisStyle-July2020}
\usepackage{cancel}
\usepackage{float}
\usepackage{cite}
\usepackage{url}
\usepackage{hyperref}
\usepackage{listings}
\usepackage[raggedright]{titlesec}
\usepackage{tabularx}
\usepackage{subcaption}
\usepackage{siunitx}
\usepackage{algpseudocode}
\sisetup{output-exponent-marker=\ensuremath{\mathrm{e}}} 
%\usepackage[none]{hyphenat}

\newcommand{\rcell}[1]{\multicolumn{1}{r}{#1}}

% ==================================
% Set up title page
% ==================================
%\includeonly{Appendix}
\title{\vspace{-1cm}Deep Reinforcement Machine Learning as a Driver of Agent Decision-Making in Agent-Based Models of Coupled Natural and Human Complex Systems}
\author{Kevin Allen Andrew}
\defensedate{July 12th, 2023}
\thesis
\masterscience
\cs
\auggrad
\advisor{Asim Zia, Ph.D.}
\chair{Scott Hamshaw, Ph.D.}
\readerone{Donna Rizzo, Ph.D.}
\readertwo{Safwan Wshah, Ph.D.}
\dean{Cynthia J. Forehand, Ph.D.}

\begin{document}

\maketitle
\pagenumbering{roman}

\begin{abstract}
	
\vspace{10mm}
Agent-based models are becoming increasingly useful in studying the behavior 
of real-world complex multi-agent systems; however, one of the outstanding 
challenges in the modeling of coupled natural and human systems is the 
dearth of techniques for creating agents that are able to learn from their
past failures and successes, as well as compounded environmental and social
uncertainties. This research has been focused on the integration of
traditional agent-based modeling with machine learning methodologies for
modeling agent decision-making and its recursive impacts on economic,
environmental, and societal outcomes, feeding into the dynamic co-evolution of
the coupled natural and human system state variables within simulated worlds,
resulting in the development of two models incorporating and exploring the use
of deep reinforcement machine learning as a driver for decision-policy making
in agent-based models.

The first of these models is a model of agricultural land use and the adoption
of agricultural best-management practices by farmers in response to ecological
and economic scenarios as a result of municipal regulation and variance in the
occurrence of extreme weather events. The primary study area used for the
model is a region of the Missiquoi Bay Area of Lake Champlain in Vermont,
containing 480 farmer agents corresponding to agricultural land parcels within
the region. A parameter sweep and sensitivity analysis on model
hyperparameters was conducted to explore the effects of changes to agent
calibration and training on agent decision-making and model performance.

The second model expands upon the scope of the first, including forester
agents and commercial and residential urban agents within a larger region of
the Lake Champlain Basin of Vermont. Additionally, the impacts of agent
decision-making take place on the simulated landscape, resulting in gradual
land cover change over time. Land cover data from the United States Geological
Survey's National Land Cover Database was used for initial parameterization,
calibration, and training of the model (years 2001, 2006) and model testing
(year 2011).

Results suggest that with appropriate scoping and hyperparameter selection,
the integration of deep reinforcement machine learning techniques into the
development of agent-based models can increase predictive accuracy in the
modeling of real-world phenomena; however, these gains must be weighed against
the increased technical complexity of such a model and the associated risk of 
introducing model error.

\end{abstract}

%\include{ack}

\newpage

\tableofcontents
\newpage


\listoffigures
\newpage

\listoftables
\newpage

\doublespacing

\setcounter{page}{1}
\pagenumbering{arabic}
\chapter{Review of Related Work}

\textbf{TODO}
\begin{itemize}
\item Additional to work into this section: Sutton and Barto (06); GRLA; Deep SARSA and Q (XU, CAO, CHEN et al, 2018); LOT of ABM Stuff
\item Make more readable and less jumpy
\end{itemize}

The use of agent-based models to study the behavior of human agents in complex 
systems primarily dates back to the early 1970s, with some of the first formal 
models being Schelling's dynamic model of segregation
\cite{schelling1971dynamic}, 
Reynolds' distributed herding model \cite{reynolds1987flocks}, 
and Axelrod and Hamilton's model for the iterative prisoner's dilemma 
\cite{axelrod1981evolution}.
While attempts to rationalize and describe human behavior date to antiquity, 
these models were among the first to demonstrate how reducing a complex system 
down to its elementary components and the simple rules that define it allows 
for its dynamic behaviors to be reliably, and repeatedly, observed and studied.

The study of emergent systematic behavior and large-scale system dynamics, 
as described in Anderson's \emph{More is Different} \cite{anderson1972more}, 
would quickly become known as complex systems studies.
Over the following decades, interest in the field grew, and the modeling of 
multi-agent systems became more widespread, resulting in the development of 
larger, more complex agent-based models.

While early models primarily focused on studying small homogeneous systems, 
as work continued through the late 1990s and into the new millennium, 
researchers began to model the behavior of more heterogeneous agent 
populations \cite{socsci00} 
and explore how agents behave when given cooperative, competitive, or 
organizational tasks \cite{comcol97}. 
Work from this period began to focus less on solipsistic agents with 
information only about their independent state and more on how information 
sharing and networking can affect agent behavior \cite{prietula1998simulating}.

The number of ways to define the behavior of agents within complex agent-based 
systems is myriad; 
however, some of the most common include probabilistic methods and rule-based 
approaches. 
For the majority of this project, the behavior of agents is going to be 
defined by artificial neural networks trained using deep reinforcement 
machine learning. 
Agent-based systems have previously incorporated reinforcement learning methods 
like SARSA and temporal difference learning (---, 1990); however, this project is one of the first to embed this type of neural network into agents within such a large-scale and heterogeneous model.

This specific application of reinforcement machine learning may be new, 
but its study is almost as old as the field of modern computer science. 
One of the first recorded mentions of reinforcement learning techniques for the
development of artificial intelligence is in Turing's 
\emph{Computing Machinery and Intelligence} \cite{machinery1950computing}, 
wherein he proposes that one possible way to 
construct an intelligent machine is to create a ``child machine,'' that,
through the application of punishments and rewards, 
is taught to behave such that 
``events which shortly preceded the occurrence of a punishment signal are 
unlikely to be repeated, whereas a reward signal [increases] the probability 
of repetition of the events which led up to it.''
Computational learning of this sort was studied more seriously over the 
following decade, eventually being dubbed 'reinforcement learning' in Minsky's 
Steps Towards Artificial Intelligence \cite{minsky61}. 
While many of the techniques of this era have been supplanted by newer 
methodologies, some of its key theoretical concepts became mainstays and went 
on to form the backbone of modern reinforcement learning--- perhaps most 
notably the development of temporal difference learning as described in 
Samuel's \emph{Some Studies in Machine Learning Using the Game of Checkers}
\cite{samuel1959some}.

Progress in the study of reinforcement machine learning saw little development over the following decade; however, a resurgence of interest in artificial intelligence during the 1970s revitalized the field and resulted in many new algorithms. Some of the more influential of these algorithms being the temporal difference learning algorithm (Sutton, 1988), the q-learning algorithm (Watkins, 1989), and the related SARSA algorithm (Rummery \& Niranjan, 1994) for decision-policy making. 

Notably, Sutton's temporal difference learning algorithm category 
$TD(\lambda)$, 
where the historical discounting factor $0\le\lambda\le 1$, 
is the basis for many of the techniques used in this project. 
Dayan (1992) proved that Sutton's temporal difference learning algorithm 
family converges for discrete problem spaces \cite{dayan92}; 
however, the problem remains undecidable for continuous-valued problems, 
so consideration must be taken for model hyperparameter selection.

Alongside these developments in reinforcement learning, advancements in computing machinery and the production and training of artificial neural networks helped bypass many of the previous limiting factors in the study of artificial intelligence. For example, the best method to correct neural network output had been an open question since their first use. But, the development of algorithms for the backpropagation of network error revolutionized the field (Rumelhart, Hinton, \& Williams, 1986). These methods allowed for the creation of networks that were more intricate and generalizable than ever before, and their increased performance made them a standard with derivatives still used today.

Entering the mid-to-late 1990s, development in artificial intelligence and reinforcement machine learning again began to stall. Problems like vanishing and exploding gradients within the hidden layers of networks, as well as physical limitations on the size and speed of machine memory, made the use and application of deep, large-scale neural networks infeasible for many potential use cases. Progress in the field remained incremental until the mid-2010s when advancements in GPU-enabled computing allowed for faster, more powerful, and more affordable high-performance computing to enter the mainstream. (---)

With this improvement in computing capabilities came several new reinforcement learning methods, including deep reinforcement learning, which makes use of the ability of artificial neural networks to perform function approximation to make the decision-policy for a problem space. By using deep neural networks in this way, the decision-policy table q-learning algorithms use to value decision-making in discrete problem spaces can be replaced with a neural network with a deep q-network architecture for decision-policy making in more continuous problem spaces (DQN).

DQN is a suitable algorithm for many reinforcement learning tasks, but it's not without its flaws. Overcoming its propensity towards biasing itself from outlier data early in training can be incredibly difficult. (Fujimoto, Hoof, \& Meer, 2018)

To combat some of the difficulties that can arise from using DQN, 
several additions and variations to the algorithm have been developed. 
The addition of policy gradient (O'Donoghue, Munos, Kavukcuglu, \& Mnih, 2017) 
and action replay (Zhao, Wang, Shao, \& Zhu, 2016) 
to the algorithm can help to smooth the learning curve and encourage 
additional exploration of the problem space. 
Additionally, 
combination algorithms like double deep q-learning (DDQN) \cite{ddqn16} 
and the rainbow algorithm \cite{rainbow18}
have been showing promising results; however, 
they are still fairly young algorithms and haven't been around 
long enough to do a proper meta-analysis of their reliability and 
accuracy across problem types.


\chapter{Machine Learning in Multi-Agent Systems}
\label{cha:farm}

One of the outstanding challenges in ABMs of coupled natural and human systems 
concerns the lack of ABMs to simulate agents with the ability to learn from 
their past failures or successes and environmental and social uncertainties.
\cite{sert2020segregation}

In this chapter, a method of integrating machine learning into ABMs is
resented as a potential solution to this problem using
a modeling methodology incorporating elements of dee reinforcement machine
learning with classical ABM techniques.
This methodology is then applied to a simple ABM of a coupled natural and human
system.
The results of this applicaption are then discussed.

\section{Methodology}
\label{sec:farm_methods}

\subsection{Modeling Approach}
\label{subsec:farm_methods_aroach}

The dee reinforcement machine learning methods being used in this model are
based on the dee q-learning methods developed by Hasselt, Guez, and
Silver\cite{ddqn16}, incorporating some of the alterations to action relay
and learning convergence as described in the rainbow algorithm developed by
Hesel et al.\cite{rainbow18} and integrating the episodic training structure
into the runtime execution of an agent-based model.

In this regard, each agent has two aired actor-critic neural network
architectures --- one air, which is used for `active' learning,
and a `target' air which is used for passive learning.
The active air is used to drive agent decision-making within the
current simulated model environment in any given time-step.
The target air is updated periodically with the weights of the active
network.
This transfer learning is done to prevent the network from overfitting 
to circumstance and to prevent the networks from diverging during training.

Within each air, the actor network ($\mu$) is responsible for selecting the
next action to take given the current state of the agent and the critic
network ($Q$) is responsible for estimating the value of the current state
given the current state and action.
The actor network is trained to maximize the value of the critic network,
whereas the critic network is trained to minimize the difference between its
estimated valuation for each state-action air with the valuation that
would be consistent with the rewards received from past events.

High-level diagrams of these architectures and how they interact with
agent states, $s = \left(s_1, ..., s_{|s|}\right)$, 
and actions, $a = \left(a_1, ..., a_{|a|}\right)$,
as vectorized components to produce value estimations
can be seen in Figure~\ref{fig:farm_ddqn}.

\begin{figure}
    \subcaptionbox{Actor-Critic Pair}
    {\includegraphics[width=.46\textwidth]{figure/ddqn1}}
    \hfill
    \subcaptionbox{Active-Target Pair}
    {\includegraphics[width=.46\textwidth]{figure/ddqn2}}
    \caption{Diagram of (a) the actor-critic network layout 
    and (b) the active-target transfer learning air used by
    agents}
    \label{fig:farm_ddqn}
\end{figure}

\subsection{Agent Decision-Making}
\label{subsec:farm_methods_decisions}

Agents in this model make decisions according to an internal decision-policy
function $\i(s)=a$ mapping the state of each agent to the potential actions 
that each agent can take.
In this approach, the decision-policy function is being approximated by
an artificial neural network (ANN), $\mu:S\rightarrow A$.
The input to this ANN is the state of the agent, vectorized as a 1-dimensional
array of length $W_S$.
The output of the ANN is a vector of length $W_A$ encoding the action that
the ANN has decided the agent should take.

Each time an agent needs to take an action,
it will ass its current state through the network to generate an action.
It will perform this action with some probability $1 - \epsilon$.
With probability $\epsilon$, the agent will instead take a random action.
This random action is used to encourage exploration of the state space
and to prevent the agent from getting stuck in a local optimum.

\subsection{Agent Memory}
\label{subsec:farm_methods_memory}

Agents in the model store a history of their past experience as
a series of state transition records $(s_t,a_t,r_t,s_{t+1})$.
These records are stored in a memory buffer $B$ of fixed length $N$.
When the memory buffer is full, new records overwrite the oldest records in
the buffer.
The memory buffer is used to train the agent's decision-policy and valuation.

Additionally,
agents have a built-in `forgetfulness factor', $F$,
which has been incorporated in an attempt to capture some of the behavior
patterns of human actors with imperfect memory.
This factor is a real number between 0 and 1 and is used to linearly scale the
amount of noise that is introduced into the memory record as the record
ages within a run.
An agent with $F=0$ will have perfect recall of its entire state transition
history, whereas an agent with $F=1$ will have perfect recall of its most
recent state transition with actions taken in the distant past being
completely forgotten (noise term of equal range as actual term).

\subsection{Agent Learning}
\label{subsec:farm_methods_learning}


\section{Experimental Design}
\label{sec:farm_ex}

\subsection{Simulation Environment}
\label{subsec:farm_ex_env}

In order to test this modeling methodology,
an experimental agent-based model was developed to explore the behavior
of a multi-agent system of agricultural decision-makers
and how that behavior may change in response to various external stimuli.
The real-world basis for this model is a study area in the
Missisquoi Bay Area of the Lake Champlain Basin of Vermont,
and the model is designed to represent the agricultural decision-making
processes of farmers in this area.
In articular, decisions pertaining to land use practices and
the adoption or rejection of agricultural best management practices (BMPs)
were studied.

The model was implemented using the FLAMEGPU framework (FLAMEGPU),
which is an agent-based modeling framework that allows for the
development of models that can be run on a GPU-device,
and with CUDNN (CUDNN),
a dee learning library for CUDA.
These technologies were selected \textbf{EXPAND RATIONALE}

\subsection{Agents}
\label{subsubsec:farm_ex_agent}

There are two types of agents resent in this model --- 480 farmer agents, 
corresponding to the 480 agriculturally-zoned land parcels in the Missisquoi 
Bay Area, 
and a single regulatory agent.
All agents in the model contain some internal information about their current 
state and history, a set of state-transition memories used to learn from 
experience, and a air of neural networks used to drive agent decision-making. 
As the agents make decisions over time, they gradually learn the correlation 
between the actions they take from each state using dee reinforcement machine 
learning.

The 480 farmer agents are used to model the behavior of agricultural land 
managers within the study area. 
These agents make annual decisions, once er time-step, 
about their farming practices, 
including whether they should adjust their productivity in one of four 
agricultural sectors (beef, dairy, corn, and hay) 
and whether they should implement an agricultural best management practice 
(BMP) to reduce phosphorus on their land.

Conditions that factor in as comonents of a farmer agent's state include the 
total land area the agent has devoted to cropland or pasture; 
the productivity of the agent in each of the four modeled agricultural 
industries along with their associated phosphorus byproduct productivity; 
an n-year history of the farm's profitability, storm losses, and BMP usage; 
and similar historical information from the agent's k-nearest neighboring
farmer agents.

The one municipal regulatory agent is used to model a municipal government or 
regulatory agency's behavior managing agricultural practices on the landscape 
and the local environment and the policies that guide them. 
This agent acts more slowly than the agricultural agents, 
once every five time-steps, and decides if/how it should modify its incentive 
structure --- changing its taxation rate, the subsidization given to an agent 
adopting a BMP, and the phosphorus runoff threshold at which a penalty is 
applied.

The municipal regulatory agent's state conditions include a history of 
extreme weather events in the region; and the aggregate profitability, 
phosphorus runoff, and storm loss across the last five time-steps for all 
agents.

\subsubsection{Problem Space}

The problem space for this model is the set of all possible states that the
agents can be in and the set of all actions that can be taken from those
states, detailed here.

The state factors used in the decision-making of each agent are listed
in Table~\ref{tab:farm_agents_states}.
For the farmer agents, these break down into a few main groups:
information about their own land cover,
information about their productivity in the given time-step,
a 5-year history of their own experiences,
and historical information from their 5-nearest neighbors.

\begin{table}
    \centering
    \caption{A summary of the state factors being used as input to the
    agents' ANNs. Agents may have additional properties
    (listed in Appendix~\ref{app:farm}); 
    however, these are the ones that action selection 
    is directly dependant on.}
    \label{tab:farm_agents_states}
    \begin{tabularx}{\linewidth}{XXl}
        \emph{Farmer Agent} \\
        \hline\hline
        Group & Description & Detail \\
        \hline
        \textbf{Land Cover} & Cropland & Normalized Area (sq m) \\
        & Pasture & Normalized Area (sq m) \\
        \textbf{Productivity} & Corn & See \ref{app:farm} \\
        & Hay \\
        & Dairy \\
        & Beef \\
        \textbf{History (5-year)} & Extreme Event Record & Occurrence \\
        & Financial Record & \ref{app:farm} \\
        & BMP Adoption Record & \\
        \textbf{Network Information} & Financials & Losses (1-year, 5-year) \\
        & BMP Adoption & Usage (1-year, 5-year) \\
        \hline \\[1.0em]
        \emph{Regulator Agent} \\
        \hline
        \hline
        Group & Description & \\
        \hline
        \textbf{Aggregate Data} & \multicolumn{2}{l}{BMP Adoption} \\
        & Financials & Net Profits, Losses \\
        & \multicolumn{2}{l}{P Runoff} \\
        \textbf{History} & Extreme Event Record & 5-year, 15-year \\
        & \multicolumn{2}{l}{BMP Adoption} \\
        & Financials & Net Profits, Losses \\
        & \multicolumn{2}{l}{P Runoff} \\
        \hline
    \end{tabularx}
\end{table}

The actions that the farmer agents can take are listed in
Table~\ref{tab:farm_agents_actions}.
These actions are divided into two groups:
adjusting the productivity of the agent in a given agricultural sector,
and adopting a BMP to reduce phosphorus runoff from the agent's land.

The actions that the regulator agent can take are listed in
Table~\ref{tab:farm_regulator_actions}.
These actions are divided into three groups:
adjusting the tax rate applied to the farmer agents,
adjusting the subsidy given to farmer agents adopting a BMP,
and adjusting the phosphorus runoff threshold at which a penalty is applied.

\begin{table}
    \caption{A summary of the action factors being used to drive agent
    decision-making for both types of agent resent in the model.
    Each group is an $n$-hot encoding set containing the subsequent
    possible actions. A ~\ref{app:farm}}
    \label{tab:farm_agents_actions}
    \begin{tabularx}{0.9\linewidth}{lXc}
        \emph{Farmer Agent} \\
        \hline\hline
        Group & Action & Reference Index \\
        \hline
        \textbf{BMP Usage} & Adopt BMP & $(0,0)$ \\
        & Don't Adopt BMP & $(0,1)$ \\
        \textbf{Corn Production} & Increase by $[0, S^+_c)$ & $(1,0)$ \\
        & Maintain & $(1,1)$ \\
        & Decrease by $[0, S^-_c]$ & $(1,2)$ \\
        \textbf{Hay Production} & Increase by $[0, S^+_h]$ & $(2,0)$ \\
        & Maintain & $(2,1)$ \\
        & Decrease by $[0, S^-_h)$ & $(2,2)$ \\
        \textbf{Dairy Production} & Increase by $[0, S^+_d)$ & $(3,0)$ \\
        & Maintain & $(3,1)$ \\
        & Decrease by $[0, S^-_d)$ & $(3,2)$ \\
        \textbf{Beef Production} & Increase by $[0, S^+_b)$ & $(4,0)$ \\
        & Maintain & $(4,1)$ \\
        & Decrease $[0, S^-_b)$ & $(4,2)$ \\
        \hline \\[1.0em]
        \emph{Regulator Agent} \\
        \hline
        \hline
        Group & Action & Reference Index \\
        \hline
        \textbf{Tax Rate} & Increase by $[0, T^+_g)$ & $(0, 0)$\\
        & Decrease by $[0, T^-_g)$ & $(0, 1)$ \\
        \textbf{BMP Subsidy} & Provide/Increase & $(1, 0)$ \\
        & Remove/Decrease & $(1, 1)$ \\
        \textbf{Phosphorous Threshold$\star$} & Scale & $2$ \\
        \hline
    \end{tabularx}
\end{table}

The reward value for farmer agents is a function of their net profitability
in the given year, after accounting for any storm losses and BMP adoption
costs.
The reward value for the regulatory agent is to minimize the
$R_r = \left<G_ * \sum_f{P} , G_l * \sum_f{loss}\right>$

\subsection{Hyperparameter Selection}
\label{subsec:farm_ex_hyer}

A summary of model hyperparameters is listed in Table~\ref{tab:farm_ex_hyer}.

Preliminary model runs were conducted to determine the optimal values for
the hyperparameters for machine learning within the model.
The learning hyperparameters that were varied in these preliminary runs were
the number of training episodes,
the number of steps between target network updates,
the number of inner layers in the neural networks,
the number of neurons in each of those inner layer,
the learning rate,
and the batch size.
The learning hyperparameters that were held constant were
the exploration rate at $\varepsilon = 0.1$,
the discount factor at $\gamma = 0.99$,
the learning transfer rate at $\tau = 0.001$,
the number of steps within a training episode at $N = 40$,
the relay memory size at $M = 10000$.

A summary of the final model hyerparameters are listed in
Table~\ref{tab:farm_ex_hyer}.
Network-specific parameters are listed in Table~\ref{tab:farm_ex_nets}.

\begin{table}
\centering
\caption{Hyperparameters and associated values with source or rationale
    for the agricultural land use model}
\label{tab:farm_ex_hyer}
\begin{tabular}{lll}
\hline
Parameter & Value & Source/Rationale \\
\hline
    Learning Rate ($\alpha$) & 0.00025 & \\
    Exploration Rate ($\varepsilon$) & 0.1 & \\
    Discount Factor ($\gamma$) & 0.99 & \\
    Transfer Rate ($\tau$) & 0.001 & cite \emph{Transfer learning} \\
    Relay Memory Size ($M$) & 10000 & \\
    Batch Size ($B$) & 32 & \cite{ddqn16} \\
%    Target Network Update Frequency ($f$) & & \\
    Number of Episodes ($N$) & 1000 & Testing \textbf{Expand} \\
    Number of Steps er Episode ($T$) & 40 
    & Econ model limitations \textbf{Expand} \\
\hline
\end{tabular}
\end{table}

\begin{table}
\centering
\caption{Network parameters for the ANNs used by the agents in the agricultural
land use model, where the $\mu_a$ and $Q_a$ columns correspond to the values
for the agricultural agents' networks and $\mu_r$ and $Q_r$ 
correspond to the values for the regulatory agent networks
\textbf{Include bias line?}}
\label{tab:farm_ex_nets}
\begin{tabular}{lllll}
    \hline
    Parameter & $\mu_a$ & $Q_a$ & $\mu_r$ & $Q_r$ \\
    \hline
    Input Nodes & 15 & 32 & 12 & 22 \\
    Inner Layers & 5 & 5 & 5 & 5 \\
    Inner Nodes & 10 & 16 & 10 & 16 \\
    Activation Function & ReLU & ReLU & ReLU & ReLU \\
    Connectivity & Full & Full & Full & Full \\
    Output Nodes & 17 & 1 & 5 & 1 \\
    Output Activation & $n$-hot & Linear & $n$-hot & Linear \\
    Output Groups ($n$) & 5 & --- & 3 & --- \\
    \hline
\end{tabular}
\end{table}

\subsection{Experimental Setup}
\label{subsec:farm_ex_setu}

The model was run for a variety of scenarios.
All scenarios tested the variables $BMP_e$, $\Delta EE$,
and $g$.
Then, there were two classes of test:
tests with agents with uniform memory accuracy (Table~\ref{tab:farm_ex_ar})
and tests with agents with heterogeneous memory accuracy
(Table~\ref{tab:farm_ex_mix}).

BMP Efficacy ($BMP_e$) was varied from 0.0 to 1.0 in increments of 0.1.
This parameter represents the effectiveness of BMPs in reducing nutrient
loading from agricultural fields.
A value of 0.0 indicates that BMPs have no effect on nutrient loading,
while a value of 1.0 indicates that BMPs completely eliminate nutrient loading.
This parameter was varied in order to determine the effect of BMP efficacy
on the behavior of the system.

Change in weather event frequency ($\Delta EE$) was varied from -0.2
to 0.2 in increments of 0.05.
This parameter represents the change in the frequency of extreme weather
events, such as heavy rainfall, that may be induced by climate change
compared to a historical baseline.

\textbf{Change verbiage from threshold to scale?}
The regulation change threshold ($g$) represents the maximum rate at which
the regulatory agent will adjust the regulatory environment.
Three values were tested: an aggressive threshold ($g=0.2$),
a moderate threshold ($g=0.05$), and
a restrictive case ($g=0$) for testing the model's ability to
operate in a static regulatory environment.

\textbf{Change wording: $F'$ for accuracy and $F$ for forgetfulness is bad.}
The impact of agent memory accuracy was tested for two types of agent
populations.
In uniform agent populations, all agents had the same memory recall
accuracy ($F'$), where $F'$ is the probability that a memory will be
recalled correctly.
In heterogeneous agent populations, agents had different memory recall
accuracies, 
where a proportion of agents ($P$) had accuracy $F=1$
and all other agents ($1-P$) had accuracy $F=0$.

\begin{table}
\centering
\caption{Table listing experimental parameters for uniform population runs}
\label{tab:farm_ex_ar}
\begin{tabular}{ll}
\hline
Variable & Values \\
\hline
BMP Efficacy ($BMP_e$) & 0, 0.1, 0.2, 0.3, 0.4, 0.5, 0.6, 0.7, 0.8, 0.9, 1.0 \\
Change in Event Frequency ($\Delta EE$)
    & -0.2, -0.15, -0.1, -0.05, 0.0, 0.05, 0.1, 0.15, 0.2 \\
Regulation Change Threshold ($g$)
    & 0, 0.05, 0.2 \\
Recall Accuracy ($F'$) & 0, 0.25, 0.5, 0.75, 1 \\
\hline
\end{tabular}
\end{table}

\begin{table}
\centering
\caption{Table listing experimental parameters for mixed population runs}
\label{tab:farm_ex_mix}
\begin{tabular}{ll}
\hline
Variable & Values \\
\hline
BMP Efficacy ($BMP_e$) & 0, 0.1, 0.2, 0.3, 0.4, 0.5, 0.6, 0.7, 0.8, 0.9, 1.0 \\
Weather Event Frequency ($\Delta EE$)
    & -0.2, -0.15, -0.1, -0.05, 0.0, 0.05, 0.1, 0.15, 0.2 \\
Regulation Change Threshold ($g$)
    & 0, 0.05, 0.2 \\
Population Mixing ($P$) & 0.25, 0.5, 0.75 \\
\hline
\end{tabular}
\end{table}

\section{Results}
\label{sec:farm_results}

\subsection{Model Performance}
\label{subsec:farm_results_robust}

For each model parameterization,
agents were trained for 1000 training episodes.
If more than 10\%  of the agents ($n=48$) in the model failed to converge
to a stable policy within 1000 training episodes, the model was discarded 
and retrained; however, this occurred in less than 2\% of model runs.
A lot showing the distribution of number of agents which converged
across model parameterizations is shown in Figure~\ref{fig:farm_sfc}.

\begin{figure}
\centering
\includegraphics[width=.4\textwidth]{figure/sfc}
\caption{Plot of the number of agents that converged to a stable policy
    for each parameterization of the model}
\label{fig:farm_sfc}
\end{figure}

In this training,
an agent's networks were considered to have converged if after 50
initial training episodes, 
the net change in the weights of the network during a transfer learning step
was less than $10^{-5}$.
This threshold was chosen to be small enough to ensure that the networks
had reached some stable policy, but large enough to avoid overfitting.

Models which were successfully trained and passed through this screening
were then run for 40 testing runs.

\subsection{Agent Behavior}
\label{subsec:farm_results_agents}

\begin{figure}
    \subcaptionbox{Initial State (y=2001)}{
        \includegraphics[width=0.3\linewidth]{figure/farm-sample-2001}
    }
    \subcaptionbox{End State (y=2040, g=0.05)}{
        \includegraphics[width=0.3\linewidth]{figure/farm-sample-2040-g05}
    }
    \caption{Sample model output showing the change in BMP adoption likelihood
    from a characteristic initial model state (a) to a characteristic end state (b) for a model run with a moderate regulatory change threshold ($g=0.05$).
    The color of each dot represents the likelihood that the agent will adopt
    BMPs, with green indicating a high likelihood and red indicating a low
    likelihood.}
    \label{fig:farm_mas}
\end{figure}
\subsubsection{Uniform Population Runs}

For model parameterizations with uniform agent populations,
the proportion of agents which adopted a BMP in each testing model run
was recorded and used to generate a distribution of BMP adoption rates
for each parameterization.
Summaries of the results of these runs are shown
in Figure~\ref{fig:farm_res_g00} for the case where the regulation change
threshold ($g$) was set to 0.0,
Figure~\ref{fig:farm_res_g05} for when $g$ was set to 0.05, and 
Figure~\ref{fig:farm_res_g20} for when $g$ was set to 0.2.

Some parameterizations have been omitted for readability.
The full listing of results can be found in
Appendix~\ref{app:results}.

\begin{figure}
    \subcaptionbox{$F=0$}{\includegraphics[width=.3\textwidth]{figure/g0F0}}
    \hfill
    \subcaptionbox{$F=0.5$}{\includegraphics[width=.3\textwidth]{figure/g0F05}}
    \hfill
    \subcaptionbox{$F=1.0$}{\includegraphics[width=.3\textwidth]{figure/g0F10}}
    \caption{Distribution of mean BMP adoption rate for uniform population
        runs of the agricultural land use model, where $g=0.0$,
        for (a) $F=0$, (b) $F=0.5$, and (c) $F=1.0$}
    \label{fig:farm_res_g00}
\end{figure}

\begin{figure}
    \subcaptionbox{$F=0$}{\includegraphics[width=.3\textwidth]{figure/g05F0}}
    \hfill
    \subcaptionbox{$F=0.5$}{\includegraphics[width=.3\textwidth]{figure/g05F5}}
    \hfill
    \subcaptionbox{$F=1.0$}{\includegraphics[width=.3\textwidth]{figure/g05F0}}
    \caption{Distribution of mean BMP adoption rate for uniform population
        runs of the agricultural land use model, where $g=0.05$,
        for (a) $F=0$, (b) $F=0.5$, and (c) $F=1.0$}
    \label{fig:farm_res_g05}
\end{figure}

\begin{figure}
    \subcaptionbox{$F=0$}{\includegraphics[width=.3\textwidth]{figure/g20F0}}
    \hfill
    \subcaptionbox{$F=0.5$}{\includegraphics[width=.3\textwidth]{figure/g20F05}}
    \hfill
    \subcaptionbox{$F=1.0$}{\includegraphics[width=.3\textwidth]{figure/g20F10}}
    \caption{Distribution of mean BMP adoption rate for uniform population
        runs of the agricultural land use model, where $g=0.2$,
        for (a) $F=0$, (b) $F=0.5$, and (c) $F=1.0$}
    \label{fig:farm_res_g20}
\end{figure}

\subsubsection{Mixed Population Runs}

For model parameterizations with mixed agent populations,
agents were divided into three groups:
group 1, where $F=0$ for the agent and all neighbors,
group 2, where $F=1$ for the agent and all neighbors, and
group 3, for agents with neighbors where $F=0$ and $F=1$. 
The proportion of agents in each group which adopted a BMP in each testing
model run was recorded and used to generate a distribution of BMP adoption
rates for each parameterization.
Results of one set of parameterizations of these runs are shown
in Figure~\ref{fig:farm_res_mix0} where $g=0$, $\Delta EE=0$.
The results indicate generally that this method of introducing
heterogeneity into the population can introduce variance in
agent behavior, but that it is unclear if it leads to any
of the desired emergent behavioral patterns.
Further testing, specifically targeting these types of populations
would be needed to draw stronger conclusions.

A full table listing the results for all parameterizations can
be found in Table~\ref{tab:full_res_mixed}.

\begin{figure}
    \subcaptionbox{$P=0.25$}{\includegraphics[width=.3\textwidth]{figure/g0P25}}
    \hfill
    \subcaptionbox{$P=0.5$}{\includegraphics[width=.3\textwidth]{figure/g0P50}}
    \hfill
    \subcaptionbox{$P=0.75$}{\includegraphics[width=.3\textwidth]{figure/g0P75}}
    \caption{Distribution of mean BMP adoption rate for mixed population
        runs of the agricultural land use model, where $g=0.0$,
        $\Delta EE=0$,
        for (a) $P=0.25$, (b) $P=0.5$, and (c) $P=0.75$}
    \label{fig:farm_res_mix0}
\end{figure}

\section{Discussion}
\label{sec:farm_disc}

\subsection{Sensitivity and Limitations}

The purpose of the regulatory agent was to help incentivize agent learning,
but in results with $g=0.2$, the increased variability in model performance
was high and the impact dominated all other parameters.
\textbf{Detail, Scaling in results, mean and variance, why these results
even matter}


\chapter{Increasing ABM Integration}
\label{chap:land}

\newcommand{\NSE}[1]{\text{NSE}_{\text{#1}}}

% The last chapter was about the agricultural land-use ABM
% This chapter is going to be about the land-cover transition ABM
% This model also uses deep reinforcement learning in order to train agents
% The goal of this model is to demonstrate the predictive accuracy of the model
% and to prove the viability of the methodology, that being,
% that adding learning to an ABM allows for more meaningful decision-making
% AND that the ABM part allows the learning to capture features that
% might be hard to model otherwise

The previous chapter demonstrated how an agent-based model can be
implemented with machine learning in order to induce a model of
individual behavior.
This chapter will go on to show how adding a transfer learning
calibration step, in cases where there exists some external metric
for determining the ABM's performance as a result of agent behavior,
can be integrated into this kind of ABM in order to adjust
agent training in a way that increases model performance.
This method is applied in the creation of a land-cover transition
model of a study area in the Lake Champlain Basin of Vermont.

\section{Methodology}
\label{sec:land_methods}
\label{sec:land_cal}

The methodology used here is similar to the methodology presented in
Chapter~\ref{chap:farm}, with the addition of an external
calibration step that takes place during model training.
Because there is an external method for validating the
output of the model for each training episode,
a traditional learning step can be incorporated into the
transfer learning step as a form of adaptive transfer learning,
where there base transfer rate~$\tau$ in
the transfer learning update (Eq~\ref{eq:tr})
is replaced with an adjusted transfer rate~$\tau_{adj}$
as in Equation~\ref{eq:tra}.

\begin{equation}
\label{eq:tra}
\forall\theta'_i\in\Theta'\left[
\theta'_i\leftarrow\tau_{adj}+(1-\tau_{adj})\theta'_i
\right]
\end{equation}

For a given calibration method, that is, a loss function
for overall model performance, the adjusted transfer rate
can be set for each transfer update based on the performance
of the model in each transfer episode.
This allows transfer episodes with a higher metric of model
performance to have a greater impact on the weights of the
target network architecture, whereas transfer episodes that
demonstrated poorer performance would have a lesser impact
on the target architecture.

\section{Experimental Design}
\label{sec:land_exp}

\subsection{Model Overview}

In order to test this calibration methodology and an
increased degree of integration of deep machine learning
techniques into agent-based modeling, a second experimental
ABM was developed.
This ABM is also a model of a study area within the Lake~Champlain
Basin of Vermont, but expanded to included systems outside
of the agricultural sector, as forestry and urban commercial
and residential systems are also included.
In this model, agent decision-making not only impacts their
economic state, but also impacts how the land-cover of the
land associated with each agent develops and changes over time.
The land-cover change is modeled as the stochastic byproduct
of agent action, wherein, for example, an agricultural agent
deciding to increase its productivity with regard to grazing
animals may result in an increase in its beef 
production factor~$p_b$ or a conversion of some unused forested 
land cells on the agent's property into in-use pasture land
cells.

Unlike the model described in Chapter~\ref{chap:farm},
the individual behavior of agents in this model is not its
primary area of interest.
Because real-world data exists for land-cover,
there exists an external, objective method for determining
the accuracy of model performance within a training episode.
The goal of this model is to explore the potential of using
real-world land-cover data and the calibration step 
and its adjustments to learning~transfer to steer agent learning, 
and consequently decision-making, in the direction of an overall 
increase in the predictive accuracy of the model.



\subsubsection{Representing Land Cover}

Real world land cover data for the study area was taken
for the United States Geological Survey's National Land Cover Dataset (NLCD)
for four years: 2001, 2006, 2011, and 2016.
This dataset divides the study area into a matrix of 30m by 30m
land cells which have been assigned one of 15 NLCD land cover classes.
For the purposes of this model, these 15 classes were divided into
6 cover categories: urban, forested, agricultural, barren, grassland/scrub,
and other.
These categories and their corresponding NLCD land cover classes
are listed in Table~\ref{tab:nlcd}.
A plot of the NLCD representation of the study area for NLCD year 2001
is shown in Figure~\ref{fig:land_plot}.

%% TAB: NLCD ---------------------------------------------------------------
\begin{table}
\centering
\caption{A listing of the 15 NLCD cover classes that are present in
the dataset for the study area, their NLCD encoding,
and their associated cover category within the model.}
\label{tab:nlcd}
\begin{tabular}{llc}
\hline
\hline
    Cover Category & NLCD Cover Class & NLCD Encoding \\
\hline
    Urban & Open Space Urban & 21 \\
    & Low Density Urban & 22 \\
    & Medium Density Urban & 23 \\
    & High Density Urban & 24 \\
    Forested & Deciduous Forest & 41 \\
    & Evergreen Forest & 42 \\
    & Mixed Forest & 43 \\
    Agricultural 
    & Pasture & 81 \\
    & Crops & 82\\
    Barren & Barren & 31 \\
    Grassland/Scrub & Scrub & 52 \\
    & Grassland & 71 \\
    Other & Water & 11 \\
    & Wetlands Woody & 90 \\
    & Wetlands Other & 95 \\
\hline
\end{tabular}
\end{table}

\begin{figure}
\centering
    \includegraphics[width=.7\linewidth]{figure/lc5012}
    \caption{A plot of the NLCD data for the selected study area
    showing land-cover in real-year 2001, each 30m by 30m land cell is 
    colored based on its NLCD cover class (Table~\ref{tab:nlcd}),
    with similar colors being numerically closer classes.}
    \label{fig:land_plot}
\end{figure}

The land cells, as they exist in the NLCD data, provide the initial land
cover for the model within the starting year.
These land cells have been divided into land parcels,
which are the collections of cells that an agent in the model
has control over.
The properties that constitute the land cells within this model
are listed in Table~\ref{tab:land_cells},
and a diagram showing how this data is overlaid on the NLCD
data is shown in Figure~\ref{fig:land_cells}.

%% TAB: LAND_CELLS ---------------------------------------------------------
\begin{table}
\centering
\caption{Land Cell Features}
\label{tab:land_cells}
\begin{tabular}{llll}
\hline
\hline
    Parameter & Values \\
    \hline
    Parcel ID & Parcel that this cell is a part of \\
    Land Cover Category & Agricultural, Forested, Urban, Other \\
    NLCD Cover Class & See Table~\ref{tab:nlcd} \\
    Land Usage Type & Managed/In-Use, Adjacent, Unmanaged \\
    \hline
\end{tabular}
\end{table}

\begin{figure}
    \centering
    \includegraphics[width=0.7\textwidth]{figure/land-cells}
    \caption{Diagram showing how land cells and their
    associated properties are represented within the model as
    a grid of NLCD land cover values (left), and how those values
    have land use properties mapped onto them 1-1 (right).}
    \label{fig:land_cells}
\end{figure}

Initial land-use for the cells within the model is generated via
a stochastic process.
Within the land-use initialization, all cells start with
a label of ``unmanaged'',
then a number of cells within each parcel are labeled as ``managed'',
depending on the agent type and weighted towards population centers.
Finally, all cells which are ``unmanaged'' and border a ``managed''
cell are labeled as ``adjacent''.

\subsubsection{Agents}
\label{subsubsec:land_exp_agents}

There are four types of agent present in this model:
agricultural agents, forestry agents, commercial agents,
and residential agents.
There is one agent assigned for each land-parcel in the model,
and the agent type is assigned based on the majority land-cover
of the parcel.

Agricultural agents model the behavior of farmers, herders, 
and other kinds of agricultural land managers within the study area.
They make annual decisions about their farming practices,
including whether they should change production in one of the four modeled 
agricultural industries (beef, dairy, corn, and hay) 
and whether they should implement an agricultural best management practice 
(BMP) to reduce phosphorous runoff on their land.
This agent type has a very similar implementation in this model
as was described for the agricultural model in Section~\ref{sec:farm_ex},
but with the land-cover change further broken down by land-use.
It's modified state factors are listed in Table~\ref{tab:land_farmer_state},
and the actions it can take are listed in Table~\ref{tab:land_farmer_act}.

\begin{longtable}{lcll}
    \caption
    {A summary of the state factors being used during decision-making
    for agricultural agents in the land-cover model.} 
    \label{tab:land_farmer_state}
    \\
    \hline\hline
    Group && Description & Detail \\
    \hline
    \endfirsthead
    \caption[]{(continued...)} \\ \hline\hline
    Group && Description & Detail \\
    \hline
    \endhead
    \hline
    \endfoot
    Land Cover &$c_{c,m}$& Cropland/In-use & Cell Count \\
    &$c_{c,a}$& Cropland/Adjacent & Cell Count \\
    &$c_{p,m}$& Pasture/In-use & Cell Count \\
    &$c_{p,a}$& Pasture/Adjacent & Cell Count \\
    &$c_{a,u}$& Agricultural/Unmaintained & Cell Count \\
    &$c_{o,a}$& Other/Adjacent & Cell Count \\
    Productivity & $p_c$ & Corn & \\
    & $p_h$ & Hay & \\
    & $p_d$ & Dairy & \\
    & $p_b$ & Beef & \\
    History && BMP Adoption Record & \\
     && Extreme Event Record & \\
     && Net Profits/Losses & \\
    Network Information && BMP Adoption & \\
     && Net Profits/Losses & \\
\end{longtable}

In order to prevent any direct interference with the production functions,
an additional land-use decision action was added to the list of actions
for the agricultural agent, to explicitly capture the intent to change
the scope of land-use.
This action had initially been planned for inclusion in the agricultural
model described in Chapter~\ref{chap:farm}, but the lack of real-world
land-cover data in the associated datasets prevented its implementation.

\begin{longtable}{lp{0.5\textwidth}}
\caption{A summary of the action factors being used to drive agent
    decision-making for agricultural agents in the land-cover model.}
    \label{tab:land_farmer_act}\\
\hline\hline
Group & Action  \\
\hline\endfirsthead
\caption[]{(continued...)}\\
\hline\hline
Group & Action  \\
\hline\endhead
\hline\endfoot
BMP Usage & Adopt BMP  \\
    & Don't Adopt BMP  \\
Corn Production & Increase by $[0, S^+_c)$  \\
    & Maintain  \\
    & Decrease by $[0, S^-_c)$  \\
Hay Production & Increase by $[0, S^+_h)$  \\
    & Maintain \\
    & Decrease by $[0, S^-_h)$  \\
Dairy Production & Increase by $[0, S^+_d)$  \\
    & Maintain  \\
    & Decrease by $[0, S^-_d)$  \\
Beef Production & Increase by $[0, S^+_b)$  \\
    & Maintain  \\
    & Decrease $[0, S^-_b)$  \\
Land-Use Decision & Grow \\
& Maintain \\
& Shrink \\
\end{longtable}

Here, it behaves as follows.
A decision to ``maintain'' land-use means that no land-use changes will
occur. A decision to ``grow'' land-use means that the agent will search
for up~to~2 land cells which are of land-use ``adjacent'' to convert into
land-use ``maintained'', and will adjust its land area calculations
to match. A decision to ``shrink'' land-use means that the agent will
attempt the opposite transition, converting up~to~2 land cells from
land-use ``maintained'' to land-use ``adjacent''.
In all other agent types, this behavior is implicit to the actions that
the agents take and their associated scoping.

\begin{figure}
\centering\includegraphics[width=0.7\textwidth]{figure/parcel-change}
\caption{Diagram showing how land-use, and consequently land-cover,
may change for an example 2-by-3 parcel over the course of a time-step.}
\end{figure}

Forestry agents model the behavior of loggers and other kinds of
forested land managers within the study area.
They make annual decisions about their practices
and whether to implement an advised management practice (AMP)
on their land.
The forestry agents are implemented very similarly to the
agricultural agents, but the land-cover of interest has been changed
to forested land, and the production function has been replaced with
a generalized forested productivity function.
The state factors used by the forestry agents in their decision-making
are listed in Table~\ref{tab:land_forest_state},
and the actions that these agents can take are listed
in Table~\ref{tab:land_forest_act}.

\begin{longtable}{lcll}
\caption{A summary of the state factors being used during decision-making
for forestry agents in the land-cover model.} 
\label{tab:land_forest_state}
\\
\hline\hline
Group && Description & Detail \\
\hline
\endfirsthead
\caption[]{(continued...)}
\\
\hline\hline
Group && Description & Detail \\
\hline
\endhead
\hline
\endfoot
Land Cover &$c_{f,m}$& Forested/In-use & Cell Count \\
&$c_{f,a}$& Forested/Adjacent & Cell Count \\
&$c_{f,u}$& Forested/Unmaintained & Cell Count \\
&$c_{o,a}$& Other/Adjacent & Cell Count \\
Productivity &$p_f$& Forestry & \\
History && AMP Adoption & \\
 && Extreme Event Record & \\
 && Net Profits/Losses & \\
Network Information && AMP Adoption & \\
 && Net Profits/Losses \\
\end{longtable}

\begin{longtable}{lp{0.5\textwidth}}
\caption{A summary of the action factors being used to drive agent
    decision-making for forestry agents in the land-cover model.}
\label{tab:land_forest_act} \\
\hline\hline
Group & Action   \\
\hline\endfirsthead
\hline\endfoot
AMP Usage & Adopt AMP  \\
    & Don't Adopt AMP  \\
Forestry Decision & Increase by $[0, S^+_f)$  \\
    & Maintain  \\
    & Decrease by $[0, S^-_f)$  \\
\end{longtable}

Agricultural and forestry agents are connected to and share information with
their 5-nearest neighbors of the same agent type; these
networks are static throughout each model run.
For both of these agent types, their learning reward is based on their
net profitability as was described in Chapter~\ref{chap:farm}.

Commercial agents model the behavior of shops, factories, offices, 
and other kinds of commercial land-holders within the study area. 
They make decisions bi/trimonthly about their workforce, 
including their available jobs and the associated salaries.
Byproducts of their actions impact the density and sprawl of urban
land cover on the landscape.
The state factors that it uses in decision-making are listed
in Table~\ref{tab:land_com_state},
and the actions they can take are listed
in Table~\ref{tab:land_com_act}.

\begin{longtable}{lp{0.5\textwidth}}
\caption{A summary of the state factors being used during decision-making
for commercial agents in the land-cover model.}
\label{tab:land_com_state}\\\hline\hline
Group & Description  \\\hline\endfirsthead
\hline\endfoot
Financial Status & \\
Employees & Capacity \\
    & Actual \\
    & Utilization \\
\end{longtable}

\begin{longtable}{lp{0.5\textwidth}}
\caption{A summary of the action factors being used to drive agent
    decision-making for commercial agents in the land-cover model.} 
    \label{tab:land_com_act}\\
\hline\hline
Group & Action  \\
\hline\endfirsthead
\caption[]{(continued...)}\\
\hline\hline
Group & Action \\
\hline\endhead
\hline\endfoot
Business Capacity & Decrease  \\
& Maintain  \\
& Increase  \\
Employees & Fire  \\
& Maintain  \\
& Hire  \\
\end{longtable}

Commercial agents use a simple compound reward function for their
performance, with a `living reward'~$t_l$, which grows as the number of 
time steps the commercial agent has gone without declaring bankruptcy
/ employee count becoming zero,
and the realized utilization~$U_e$ of their land,
such that the reward~$R_c=t_l*U_e$.

Residential agents model the behavior of renters and landowners within 
the study area. 
They make two decisions annually: whether to attempt a job change 
and whether to try to move houses. 
Household satisfaction, and their reward value~$R_r$,
is valued as a combination of financial stability 
and mental satisfaction. 
Each household earns wages provided by a commercial agent --- 
these wages are determined by a stochastic process and can be adjusted by 
the job over time.
The decisions of these agents do not directly impact land cover change
on their associated parcel, but land cover can transition within their
parcel as a result of the decisions of other agents.

\begin{longtable}{lp{0.5\textwidth}}
\caption{A summary of the state factors being used during decision-making
for residential agents in the land-cover model.}
\label{tab:land_res_state} \\ \hline \hline
Group & Description \\ \hline \endfirsthead
\hline \endfoot
Financial Status & \\
Length in State & \\
Household Budget & Monthly \\
Household Budget & Yearly \\
Failed Action Count & Consecutive \\
\end{longtable}

\begin{longtable}{lp{0.5\textwidth}}
\caption{A summary of the action factors being used to drive agent
    decision-making for residential agents in the land-cover model.} \\
\hline\hline
Group & Action  \\
\hline\endfirsthead
\caption[]{(continued...)} \\
\hline\hline
Group & Action  \\
\hline\endhead
\hline\endfoot
Employment & Search for new job  \\
 & Keep current job  \\
Housing & Search for new housing  \\
& Keep same housing \\
\end{longtable}

The reward function used by residential agents is similar
to that used by commercial agents, but the living reward is
scaled by the agent's financial status.

Commercial and residential agents exist in a bipartite network with
one another.
This network is initialized via a stochastic process and is
updated as agents make decisions.
This process is detailed in Appendix~\ref{app:land},
alongside other details of agent initialization.

The learning architecture for agents of each type is
listed in Table~\ref{tab:land_anns}.
Similarly to the agent-networks described in the
model in Chapter~\ref{chap:farm},
all networks are fully-connected,
use ReLU activation and He Initialization,
and use $n$-hot output to encode their selected action.
Additionally, all target architectures are initially identical
to the active network architectures as specified below.

\begin{table}[h]
\centering
\caption{Network parameters for the ANNs used by agents in each class
    for the land cover model}
\label{tab:land_anns}
    \begin{tabular}{@{\extracolsep{4pt}}lp{.05\linewidth}>{\centering}p{.05\linewidth}>{\centering}p{.05\linewidth}>{\centering}p{.05\linewidth}>{\centering}p{.05\linewidth}>{\centering}p{.05\linewidth}>{\centering}p{.05\linewidth}cc@{}}
\hline
\hline
\multirow{2}{*}{Parameter} 
    & \multicolumn{2}{c}{Agricultural} & \multicolumn{2}{c}{Forestry} 
    & \multicolumn{2}{c}{Commercial} & \multicolumn{2}{c}{Residential} \\
    \cline{2-3}\cline{4-5}\cline{6-7}\cline{8-9}
 & $\mu$ & $Q$ & $\mu$ & $Q$ & $\mu$ & $Q$ & $\mu$ & $Q$  \\
\hline
Input Nodes  & 15 & 32 & 10 & 15 & 4 & 10 & 5 & 9 \\
Inner Layers  & 4 & 3 & 4 & 3 & 2 & 2 & 2 & 2 \\
Inner Nodes  & 10 & 16 & 7 & 7 & 5 & 5 & 4 & 5 \\
Output Nodes  & 17 & 1 & 5 & 1 & 6 & 1 & 4 & 1 \\
\hline
\end{tabular}
\end{table}

\subsubsection{Model Hyperparameters}

A summary of the fixed hyperparameters across all runs of this
model are listed in Table~\ref{tab:land_hyper}.
Many of the parameters are taken directly from the
agricultural model described in Chapter~\ref{chap:farm},
with the economic production functions taking from the
corresponding model years and the weather generation
submodel using the baseline case of $\Delta EE = 0$.

The length of the training episode was set to 60 time-steps,
or 5 model-years, to match the granularity of the real-world
NLCD~datasets.
Preliminary model runs for determining appropriate fixed
model hyperparameters were done with the land-cover data
from NLCD year 2001 as input data, comparing
the data from NLCD year 2006 with the model output year 2006.

\begin{longtable}{lcll}
\caption{Fixed hyperparameters and their associated values
for the land-cover change model.}
\label{tab:land_hyper}\\\hline\hline
Parameter & & Value \\
\hline\endfirsthead
\caption[]{(continued...)}\\\hline\hline
Parameter & & Value \\
\hline\endhead
\hline\endfoot
Learning Rate & $\alpha$ & 0.00025 \\
Exploration Rate & $\epsilon$ & 0.1 \\
Base Transfer Rate & $\tau$ & 0.001 \\
Transition Memory Size & $M$ & 10000 \\
Number of Steps per Episode & $T$ & 60 \\
\end{longtable}

\subsubsection{Evaluating Model Performance}

For this model, the Nash-Sutcliffe efficiency index (NSE) was used
to evaluate the goodness-of-fit of model output at the end of each
training episode.
The NSE is a measure of the relative magnitude of the residual variance of
modeled data compared to the residual variance of the observed data.
The value of the index ranges from~$-\infty$~to~1,
where a score of 1 indicates a perfect fit,
a score of 0 indicates that the model's fit is no better than the mean
of the observed data,
and a score less than 0 indicates that the mean of the observed data
is a better predictor than the model.

This index was calculated in relation to three forms of model performance.
The first is the ability of the model to appropriately predict
the proportional coverage of each land cover type in the target year,
$\text{NSE}_\text{plc}$.
The second is the ability of the model to appropriately predict
the categorical transitions of land cover types from the start year to
the target year,
$\text{NSE}_\text{cat}$.
The third is the ability of the model to appropriately predict
the absolute transitions of land cover types from the start year to
the target year,
$\text{NSE}_\text{abs}$.
These indices are detailed below.

A majority of land cells in the study area do not transition land cover
between the start year and target year ($92.6\%, n=69124$),
which would heavily bias any analysis of model performance.
Therefore, these NSE indices were only calculated for those cells
that transitioned land cover.

The NSE measure of proportional coverage ($\text{NSE}_\text{plc}$)
is shown in Equation~\ref{eq:nse_plc},
where~$P_B$ represents the observed proportional coverage of land cover
category~$B$ in the target year,
$\hat{P_B}$~represents the simulated proportional coverage of land cover
category~$B$ in the target year,
and~$\bar{P_B}$ represents the mean observed proportional coverage of
land cover category~$B$ in the target year.

\begin{equation}
    \label{eq:nse_plc}
    \text{NSE}_{\text{plc}} 
    = \frac{\sum_B\left(P_B - \hat{P_B}\right)^2}
        {\sum_B\left(P_B-\bar{P_B}\right)^2}
\end{equation}

The NSE measure of categorical land cover transitions 
($\text{NSE}_\text{cat}$), shown in Equation~\ref{eq:nse_cat},
where $\Delta_{A,B}$ represents the number of observed transitions
from land cover category $A$ in the starting year to land cover
category $B$ in the target year,
for the overarching categories shown in Table~\ref{tab:nlcd},
where $\widehat{\Delta_{A,B}}$ represents 
the number of simulated transitions from $A$ to $B$,
and where $\overline{\Delta_{A,B}}$ represents the mean observed number of
transitions from $A$ to $B$.

\begin{equation}
    \label{eq:nse_cat}
    \text{NSE}_{\text{cat}}
    = \frac{\sum_{A,B}\left(\Delta_{A,B}-\widehat{\Delta_{A,B}}\right)^2}
        {\sum_{A,B}\left(\Delta_{A,B}-\overline{\Delta_{A,B}}\right)^2},
    A \ne B
\end{equation}

The NSE measure for absolute land cover transition 
($\text{NSE}_\text{act}$), shown in Equation~\ref{eq:nse_act},
is very similar to the calculation of $\text{NSE}_\text{cat}$,
except that it is calculated for the absolute land cover class of each
land cell and not just its categorical class.
This index is a stricter metric than the previous two indices,
as, for example,
a modeled transition from cropland to deciduous forest $\delta_{82,41}$
compared to an observed transition to mixed forest $\delta_{82,43}$,
which would be labeled ``correct'' according to $\NSE{cat}$
would be considered a misclassification according to $\NSE{abs}$.

\begin{equation}
    \label{eq:nse_act}
    \text{NSE}_{\text{abs}} 
    =
    \frac{\sum_{a,b}\left(\Delta_{a,b} - \widehat{\Delta_{a,b}}\right)^2}
        {\sum_{a,b}\left(\Delta_{a,b} - \overline{\Delta_{a,b}}\right)^2}
    ,
    a\ne b
\end{equation}

For the purposes of model calibration,
as described in Section~\ref{sec:land_cal},
the model was calibrated to maximize the result of $\NSE{cat}$
for the transition from the starting year 2006,
to the observed and modeled year 2011.

\subsubsection{Execution Overview}

A high-level overview of model execution,
showing the main training loop with the transfer calibration step
and the final testing loop, is shown 
in Figure~\ref{fig:land_exec_training}.
A more-detailed overview of model execution, specifically focused on
the behavior and training and agents is
shown in Figure~\ref{fig:land_exec_2}.

\begin{figure}
\centering
    \includegraphics[width=0.7\linewidth]{figure/land-flow}
    \caption{Flowchart showing a high-level overview of model
    execution between the main training/calibration loop (left),
    and the final testing runtime loop (right)}
    \label{fig:land_exec_training}
\end{figure}

\begin{figure}
\centering
\includegraphics[width=0.7\textwidth]{figure/flowchart2.png}
\caption{Flowchart demonstrating the overall execution of the agent-based model
    and its coupling with the machine learning processes}
\label{fig:land_exec_2}
\end{figure}

%%%%%%%%%%%%%%%%%%%%%%%%%%%%%%%%%%%%%%%%%%%%%%%%%%%%%%%%%%%%%%%%%%%%%%%%%%%
%%%%%%%%%%%%%%%%%%%%%%%%%%%%%%%%%%%%%%%%%%%%%%%%%%%%%%%%%%%%%%%%%%%%%%%%%%%

\subsection{Experimental Setup}
\label{subsec:land_exp_setup}

In order to avoid some of the high levels of variance seen
in the experimental runs of the agricultural model and that model's sensitivity
to agent paramterization, these experimental runs were limited in scope
to a subset of model hyperparameters that relate to agent memory
and foresight, listed in Table~\ref{tab:land_exp}.
Each combination of parameterizations was tested for 40 model runs.

Replay batch size~$B$ was tested for the sizes 8, 16, 32, and~64.
This value determines how many state transition records are sampled
during the action-replay and policy gradient learning steps.
A higher number of records sampled increases the smoothness of
the gradient being sampled, at the cost of extra computation.

The discount factor~$\gamma$ was tested for the values $0.5, 0.9$,
and $0.99$.
This factor is used in the Q-learning update (Eq~\ref{eq:q_update})
to discount the expected rewards at a future time-step.
A lower value of~$\gamma$ indicates a lower value of the
reward in step $t+1$ than the current reward in step $t$,
which compounds multiplicably with each additional time-step forward.

The recall accuracy~$F'$, is the notational inverse of the
forgetfulness factor described in Section~\ref{subsec:farm_methods_memory},
where $F'=(1-F)$ and consequently a recall accuracy of~1 indicates agents
with~0 forgetfulness, that is, perfect record recall.

\begin{table}
\caption{Experimental parameters that were tested in experimental runs
of the land-cover transition model}
\label{tab:land_exp}
\centering
\begin{tabular}{lcl}
\hline
\hline
    Variable && Values \\
    \hline
    Batch Size & $B$ & 8, 16, 32, 64 \\
    Discount Factor & $\gamma$ & 0.5, 0.9, 0.99 \\
    Recall Accuracy & $F'$ & 0, 0.25, 0.5, 0.75, 1 \\
    \hline
\end{tabular}
\end{table}

% For the purposes of analyzing the results of this model,
% where these types of transitions occur,
% they are treated as neither a successful nor an unsuccessful measure
% of model performance.
% Instead, they are treated as a separate neutral measure.
% 
% This limitation of the model is discussed in greater detail
% in Section~\ref{sec:land_disc}.
% 
% 
% For many of these missing transitions,
% there is a lack of any substantive literature supporting a predictive model 
% that is generalizable enough to meaningfully incorporate into the ABM.


\section{Results}
\label{sec:land_results}

\subsection{Model Performance}
\label{subsec:land_results_performance}

Model performance under each experimental parameterization was
evaluated by comparing the land cover in model year 2016 to
the recorded/observed land cover for the study area for real year 2016.

The maximum NSE values seen during testing runs were 
$\text{NSE}_\text{plc} = 0.84$,
$\text{NSE}_\text{cat} = 0.76$, and
$\text{NSE}_\text{act} = 0.64$.
The sensitivity of these indices to each of the experimental parameters
was evaluated by calculated the mean and variance of the indices
under each parameterization across all 40 runs.
These results have been plotted in Figure~\ref{fig:land_nse}.

\begin{figure}
    \centering
    \subcaptionbox{$\text{NSE}_\text{plc}$}{\includegraphics[width=0.3\textwidth]{figure/nscplc}}
    \hfill
    \subcaptionbox{$\text{NSE}_\text{cat}$}{\includegraphics[width=0.3\textwidth]{figure/nsccat}}
    \hfill
    \subcaptionbox{$\text{NSE}_\text{act}$}{\includegraphics[width=0.3\textwidth]{figure/nscact}}
    \caption{NSE index sensitivity for each index showing the variance in
    model classification accuracy by each metric under different model
    parameterizations.}
    \label{fig:land_nse}
\end{figure}

In order to better understand the output of the model and where misclassification was
occurring within testing runs, the land-cover transition matrices were analyzed.
Categorical cover change had been used as the calibration function for the model,
so land cover transitions were divided into three primary categories: urban~$U$,
forested~$F$, and agricultural~$A$, and an other~$O$ category for all other land-cover categories;
and instances where both the real and modeled data
had a matching transition from a land-cover category to itself were removed.

For the purposes of this discussion,
the two land-cover transition matrices that will be compared against the actual
land-cover transition are the ``best fit'' model ($\NSE{cat}=0.76$)
and the ``best median'' model, the median performer for the 
highest scoring parameterization during testing ($\NSE{cat}=0.68$, $\gamma=0.99$, $B=64$, $F'=1$).
Arrow plots, summarizing how land-cover transitions compared between these models,
are shown in Figure~\ref{fig:arrows},
with the associated confusion matrices listed in Table~\ref{tab:confusion}

\begin{figure}
\centering
\subcaptionbox{Observed Cover Change}{\includegraphics[width=0.3\textwidth]{figure/cat_obs}}
\hfill
\subcaptionbox{``Best Fit'' Cover Change}{\includegraphics[width=0.3\textwidth]{figure/cat_max}}
\hfill
\subcaptionbox{Best Median Change}{\includegraphics[width=0.3\textwidth]{figure/cat_med}}
\caption{Arrow plots showing a comparison of the categorical transitions between
the three main land-cover categories between
(a) the real world data, and (b) the ``best-fit'' model, and (c) the ``best median'' model.
Arrow width between NLCD-11 and 2016 is proportional to the number of transitions between the
connected categories.}
\label{fig:arrows}
\end{figure}

\begin{table}
    \centering
    \caption{Confusion matrix comparing the resulting categorical transitions from the NLCD data to the modeled transitions
    for both the best fit, and best performing parameterization.
    Because the initial cover category for both transitions are definitionally equal,
    it has been omitted from the header row.
    The transition $\Delta_{*,O}$ represents a transition outside one of the three major cover categories.}
    \label{tab:confusion}
    \begin{tabular}{@{\extracolsep{4pt}}c|cccc|cccc@{}}
    & \multicolumn{4}{c}{``Best Fit'' 2016} & \multicolumn{4}{c}{Best Median 2016} \\ \cline{2-5}\cline{6-9}
    NLCD 2016 & $\Delta_{*,U}$ & $\Delta_{*,F}$ & $\Delta_{*,A}$ & $\Delta_{*,O}$ 
    & $\Delta_{*,U}$ & $\Delta_{*,F}$ & $\Delta_{*,A}$ & $\Delta_{*,O}$ \\ \hline
    $\Delta_{U,U}$ & --- & 0 & 0 & 0 & --- & 0 & 0 & 0 \\
    $\Delta_{U,F}$ & 0 & 0 & 0 & 0 & 0 & 0 & 0 & 0 \\
    $\Delta_{U,A}$ & 0 & 0 & 0 & 0 & 0 & 0 & 0 & 0 \\
    $\Delta_{U,O}$ & 0 & 0 & 0 & 0 & 0 & 0 & 0 & 0 \\ \hline
    $\Delta_{F,U}$ & 7 & 0 & 0 & 0 & 7 & 0 & 0 & 0 \\
    $\Delta_{F,F}$ & 1 & --- & 5 & 0 & 0 & --- & 7 & 0 \\
    $\Delta_{F,A}$ & 0 & 0 & 0 & 0 & 0 & 0 & 0 & 0\\
    $\Delta_{F,O}$ & 0 & 7 & 0 & 36 & 3 & 5 & 5 & 30 \\ \hline
    $\Delta_{A,U}$ & 11 & 0 & 0 & 0 & 11 & 0 & 0 & 0 \\
    $\Delta_{A,F}$ & 0 & 0 & 0 & 0 & 0 & 0 & 0 & 0 \\
    $\Delta_{A,A}$ & 7 & 1 & --- & 0 & 3 & 14 & --- & 0 \\
    $\Delta_{A,O}$ & 0 & 4 & 4 & 3  & 1 & 0 & 10 & 0 \\
    \end{tabular}
\end{table}

Within the study area, cells with urban land-cover did not transition into any 
other category of land-cover during the study period.
Forested and agricultural cells which did transition categories,
primarily developed into urban land-cover, followed by
transitions into other categorizations.
These forested and agricultural transitions are the primary source of
the classification error within these models and are discussed further
is the following section.

\section{Discussion}
\label{sec:land_disc}

The experimental parameters which were tested each had a different degree
of impact on model performance.
The recall factor~$F'$, had the strongest impact on model performance,
with values less than 0.75 resulting in model runs that were worse performers
than the probabilistic average for $\NSE{act}$,
and with a value of 0.5 resulting in worse performance than average for
all 3~indices.
In the case of $\NSE{act}$, there is a noticeable drop in model performance
for the case of $\gamma=0.99$ compared to $\gamma=0.9$; however, 
this change is not large enough to be statistically significant when
the variance in model performance for the case of $\gamma=0.9$ is considered.

Looking at the transition plots and confusion matrices for some of the
higher accuracy model runs,
the largest source of error within the tested models came from land-cover
transitions between the agricultural and forested land-cover categories.
These transitions, like most land cover transitions between cover categories,
were poorly represented within the calibration data, as land-cover transition
is a relatively slow process that even the 5-year interval of the NLCD dataset
has difficulty capturing.
Similarly, the highest performing models tended to overclassify urbanization land-cover
changes, which were the most represented cross-category land-cover change 
in the calibration data set.

The categorical classification into the group other~$O$ was the most accurate cross-category
predictor --- in particular, categorical transition into the Grassland/Scrub cover-category.
These transitions are present in the calibration and testing datasets,
and frequently occur in proximity to changes in agricultural and forested land-cover changes.
It is possible that this is related to the high confusion in the land-cover transitions
between agricultural and forested land-cover regions, but that level of testing would
require land-cover datasets for similar regions with similar land-cover transition patterns,
which were outside of the scope of this modeling effort.

If this work were to be continued, it would be interesting to see how this type of
integrated land-cover transition model would perform over a longer period of time
or wihtin a larger study area, so the number of and type of land-cover transitions
occurring within the calibration dataset would be more representative of the
land-cover transitions occurring within the study period.


\newpage

\setstretch{1.0}
\nocite{*}
\bibliography{bibfile}
\bibliographystyle{unsrt}

\addtocontents{toc}{\protect\setcounter{tocdepth}{1}}
\appendix
\chapter{Farm Model ODD+D Document}
\label{app:farm}

text

\chapter{Land-Cover Model Design Overview}
\label{app:land}

This appendix contains an overview of the design of the land-cover transition
model described in Chapter~\ref{chap:land}.
It generally adheres to the technical principles of the ODD+D format,
as shown in Appendix~\ref{app:farm},
but is focused on where the specification provides details that do not appear
in the main text of this thesis.

\section{Overview}
\subsection{Purpose}
The purpose of this model is to explore potential changes to land cover as a result of human behavior as it develops in response to projected climatological, economic, and social scenarios within study areas of the Lake Champlain Basin of Vermont.
Four types of human agents are present in this model; these agents represent some of the various types of people who make decisions that can change land cover on the landscape. Agents received input from their environment, including inter-agent communication and stochastic environmental factors (f.x. simulated extreme weather events). Agents made decisions as frequently as once per model month, and the decision policy guiding their decision-making was trained using deep reinforcement machine learning.

\subsection{Entities, State Variables, and Scales}

\subsubsection{Study Area}
The study area being used as the basis of this model is a subsection 
of the Lake Champlain Basin of Vermont. 
Each parcel within the study area is treated as an agent within the model.
A map of the study area and its initial land-cover is
shown in Figure~\ref{fig:land_cells}.

\subsubsection{Agents}

There are four types of human agents present in this model --- agricultural,
commercial, residential, and forester. 
These agents represent various types of landowners/managers within each 
study area who are able to make decisions that can affect land-use 
and land-cover. 
They make these decisions based on both their material state 
and their perceived mental and financial state. 
Agent behavior is trained using deep reinforcement machine learning, 
which provides each agent with a decision-policy that guides their 
decision-making during test model runs.

The state variables that are present in every human agent in the model are outlined in Table~\ref{tab:land_state_share}

\begin{longtable}{lcp{0.5\linewidth}l}
\caption{Table of all properties that are shared amongst all human
agents in the land-cover transition model.}
\label{tab:land_state_share} \\
\hline\hline
Name && Description & Data type \\
\hline\endfirsthead
\caption[]{(continued...)}\\
\hline\hline
Name && Description & Data type \\
\hline\endhead
\hline\endfoot
Agent ID && Unique identifier for this agent & \tt{uint} \\
Parcel ID && Identifier of underlying land cell parcel & \tt{uint} \\
Agent Status 
&& Class-dependent internal status & \tt{enum} \\
\multicolumn{4}{l}{Neural Networks} \\
\multicolumn{1}{r}{Actor Network} & $\Theta_\mu$ & Network Weights & \tt{float[][]} \\
\multicolumn{1}{r}{Critic Network} & $\Theta_Q$ & Network Weights & \tt{float[][]} \\
\multicolumn{1}{r}{Target Actor} & $\Theta_{\mu'}$ & Network Weights & \tt{float[][]} \\
\multicolumn{1}{r}{Target Critic} & $\Theta_{Q'}$ & Network Weights & \tt{float[][]} \\
\end{longtable}

\subsubsection*{Agricultural Agents}
Agricultural agents model the behavior of farmers, herders, and other kinds of agricultural land managers within each study area. They make annual decisions about their farming practices, including whether they should change production in one of the four modeled agricultural industries (beef, dairy, corn, and hay) and whether they should implement an agricultural best management practice (BMP) to reduce phosphorous runoff on their land.
The additional state variables that are found in every agricultural agent in the model are outlined in Table~\ref{tab:land_farm_state}.

\begin{longtable}{lcp{.4\linewidth}l}
\caption{Table of all state properties of agricultural agents and their
associated data type for agricultural agents in the land-cover transition model.}
\label{tab:land_farm_state} \\
\hline \hline
Name && Description & Data Type \\
\hline
\endfirsthead
\caption[]{(continued...)}\\
\hline\hline
Name && Description & Data Type \\
\hline\endhead
\hline\endfoot
\multicolumn{4}{l}{Land Parcel Data (sq~km)} \\
\multicolumn{1}{r}{Crop Land Area} & $A_c$ & Land devoted to growing crops & \tt{float} \\
\multicolumn{1}{r}{Pasture Land Area} & $A_p$ & Land devoted to grazing animals & \tt{float} \\
\multicolumn{1}{r}{Total Land Area} & $A_{tot}$ & Total land in parcel & \tt{float} \\
\multicolumn{4}{l}{Land Cover (Cell Count)} \\
& $c_{c,m}$ & Cropland/In-use & \tt{uint} \\
& $c_{c,a}$ & Cropland/Adjacent & \tt{uint} \\
& $c_{p,m}$ & Pasture/In-use & \tt{uint} \\
& $c_{p,a}$ & Pasture/Adjacent & \tt{uint} \\
& $c_{a,u}$ & Agricultural/Unmaintained & \tt{uint} \\
& $c_{o,a}$ & Other/Adjacent Cell Count & \tt{uint} \\
Productivity \\
\multicolumn{1}{r}{Corn} & $p_c$ & Corn production factor & \tt{float} \\
\multicolumn{1}{r}{Hay} & $p_h$ & Hay production factor & \tt{float} \\
\multicolumn{1}{r}{Beef} & $p_b$ & Beef production factor & \tt{float} \\
\multicolumn{1}{r}{Dairy} & $p_d$ & Dairy production factor & \tt{float} \\
\multicolumn{1}{r}{Phosphorous} & $p_{p,x}$ & Phosphorus production factors & \tt{float[3]} \\
Cows Owned &&& \tt{uint} \\
\multicolumn{4}{l}{Financial History (5-year)} \\
\multicolumn{1}{r}{Realized Net} && Net yearly production & \tt{float[5]} \\
\multicolumn{1}{r}{Expected Net} && Expected yearly production& \tt{float[5]} \\
Extreme Event History &&& \tt{bool[5]}\\
BMP Usage History && Did farm use BMP in last 5 years & \tt{bool[5]} \\
Neighbors && References to neighboring agents & \tt{uint[5]} \\
\end{longtable}

\subsubsection*{Forestry Agents}

Forestry agents model the behavior of loggers and other kinds of
forested land managers within the study area.
They make annual decisions about their practices
and whether to implement an advised management practice (AMP)
on their land.
The forestry agents are implemented very similarly to the
agricultural agents, but the land-cover of interest has been changed
to forested land, and the production function has been replaced with
a generalized forested productivity function.
The additional state factors used by the forestry agents in their decision-making
are listed in Table~\ref{tab:land_state_for}.

\begin{longtable}{lcp{.4\linewidth}l}
\caption{Table of state properties of forestry agents and their
associated data types in the land-cover transition model.}
\label{tab:land_state_for} \\
\hline \hline
Name & & Description & Data Type \\ \hline
\endfirsthead
\hline \hline
Name & & Description & Data Type \\ \hline
\endhead
\hline\endfoot
\multicolumn{4}{l}{Land Parcel Data (sq~km)} \\
\multicolumn{1}{r}{Forested Land Area} & $A_p$ & Total forested land in parcel & \tt{float} \\
\multicolumn{1}{r}{Total Land Area} & $A_{tot}$ & Total land in parcel & \tt{float} \\
\multicolumn{4}{l}{Land Cover (Cell Count)} \\
& $c_{f,m}$ & Forested/In-use & \tt{uint} \\
& $c_{f,a}$ & Forested/Adjacent & \tt{uint} \\
& $c_{f,u}$ & Forested/Unmaintained & \tt{uint} \\
& $c_{o,a}$ & Other/Adjacent Cell Count & \tt{uint} \\
\multicolumn{4}{l}{Productivity} \\
\multicolumn{1}{r}{General} & $p_f$ 
    & Relative production factor & \tt{float} \\
\multicolumn{1}{r}{Phosphorous} & $p_p$
    & Relative production factor & \tt{float} \\
\multicolumn{4}{l}{Financial History (5-year)} \\
\multicolumn{1}{r}{Realized Net} && Net yearly production & \tt{float[5]} \\
\multicolumn{1}{r}{Expected Net} && Expected yearly production & \tt{float[5]} \\
Extreme Event History &&& \tt{bool[5]}\\
AMP Usage History && Did agent use AMP in last 5 years & \tt{bool[5]}\\
Neighbors & & References to neighboring agents & \tt{uint[5]} \\
\end{longtable}

\subsubsection*{Commercial Agents}

Commercial agents model the behavior of shops, factories, offices, 
and other kinds of commercial land-holders within the study area. 
They make decisions bi/trimonthly about their workforce, 
including their available jobs and the associated salaries.
Byproducts of their actions impact the density and sprawl of urban
land cover on the landscape.
The additional state factors that it uses in decision-making are listed
in Table~\ref{tab:land_state_com}.

\begin{longtable}{lcp{.45\linewidth}l}
\caption{Table of state properties of commercial agents and their
associated data types in the land-cover transition model.}
\label{tab:land_state_com} \\
\hline \hline
Name & & Description & Data Type \\ \hline
\endfirsthead
\hline \hline
Name & & Description & Data Type \\ \hline
\endhead
\hline\endfoot
Days Operational && Number of days operating & \tt{uint}\\
Employee Count && Current number of employed agents & \tt{uint}\\
Employee Capacity && Maximum number of employed agents & \tt{uint}\\
Employee IDs && IDs of employed agents & \tt{uint*}\\
Employee Salaries && Salaries of employed agents & \tt{float*}\\
Total Pay && Total salaried paid to all employees & \tt{float} \\
Budget && Total monthly budget & \tt{float} \\
\end{longtable}

\subsubsection*{Residential Agents}

Residential agents model the behavior of renters and landowners within 
the study area. 
They make two decisions annually: whether to attempt a job change 
and whether to try to move houses. 
Household satisfaction, and their reward value~$R_r$,
is valued as a combination of financial stability 
and mental satisfaction. 
Each household earns wages provided by a commercial agent --- 
these wages are determined by a stochastic process and can be adjusted by 
the job over time.
The decisions of these agents do not directly impact land cover change
on their associated parcel, but land cover can transition within their
parcel as a result of the decisions of other agents.

\begin{longtable}{lcp{.5\linewidth}l}
\caption{Table of state properties of residential agents and their
associated data types in the land-cover transition model.}
\label{tab:land_state_res} \\
\hline \hline
Name & & Description & Data Type \\ \hline
\endfirsthead
\hline \hline
Name & & Description & Data Type \\ \hline
\endhead
\hline\endfoot
Employer ID && ID of current employer & \tt{uint} \\
Housing Costs && Total monthly cost of living & \tt{float} \\
Salary && Total monthly income from employer & \tt{float} \\
Monthly Budget && Net income over past 1 month & \tt{float} \\
Yearly Budget && Net income over past 12 months & \tt{float[12]} \\
Time in State && Number of time steps with current `mood' & \tt{uint} \\
Failed Action Count && Number of consecutive failed actions & \tt{uint} \\
\end{longtable}

\subsection{Process Overview and Scheduling}

Each time-step within the model is representative of one modeled month.
Each model year (12 time-steps), the number of extreme weather events for the year are
generated and the agricultural and forestry agents act.

The commercial agent has two primary decision-vectors,
the decision to increase or decrease overall productivity is resolved trimonthly (3 time-steps),
while the decision to hire/fire staff is resolved bimonthly (2 time-steps).
At the end of each model year (12 time-steps), the commercial agent will increase the
salary of all employed residential agents by 10\%.

A commercial agent which loses its last employee,
by firing or quitting, will enter the ``bankrupt'' state at the end of
the occurent time-step, and will not be replaced with a ``new'' commercial agent until
3 time-steps have passed.
This new agent has a period of 3--6 time-steps to acquire its first employee before it
can bankrupt again.

The residential agent acts monthly (1 time-step); however, its actions are dependant
on the behavior of other agents in the model --- f.x. a residential agent cannot move
into an occupied house regardless of its price.
If a residential agent intends to leaves its job, house, or the entire system,
this action cannot be blocked, it is only additive actions which face this restriction.

A new residential agent may only enter the system if there is an available residential
parcel with an occupation status of ``vacant'' or ``for-sale''
and the total number of residential agents within the system does not match nor exceed
the current residential capacity.
This new residential agent will be added to the system at the start of the following
time-step.

\section{Initialization}

\subsection{Agent Connections}

Agricultural and forestry agents are all connected to their
5-nearest agents of the same type.
Since these agents do not move on the landscape, these networks are
constant throughout each model run.

Commercial and residential agents exist within a bipartite network.
During model initialization, each commercial agent starts with an
employment capacity of 10.
At model start, 90\% of residential agents are assigned jobs
selected with uniform likelihood from that available commercial capacity.
The initial employment capacity within the selected study area ensures that
there will always be more initial capacity than 90\% of the initial residential capacity.
These networks are dynamic and change as agents make decisions throughout each model run.

\subsection{Initial Agent State}

Many of these values are selected from triangular distributions
which was parameterized according to the calibration of the
underlying economic model.
Triangular distributions, as described in Section~\ref{sec:farmer_init},
are used for the initialization of many parameters.
Neural networks had weights initialized according to the He~initialization
algorithm.~\cite{he2004initialization}.

\begin{longtable}{lcll}
\hline\hline
Value & & Initialization & Type \\
\hline
\endhead
\hline\endfoot
Agent ID & & Assigned sequentially from 0 & \tt{uint} \\
Parcel ID & & Read from Input Data & \tt{uint} \\
Actor-Network Weights & $\Theta_\mu$ & He() & \tt{float[][]} \\
Critic-Network Weights & $\Theta_Q$ & He() & \tt{float[][]} \\
Target-Actor Weights & $\Theta_{\mu'}$ 
    & Copied from $\Theta_\mu$ & \tt{float[][]} \\
Target-Critic Weights & $\Theta_{Q'}$ 
    & Copied from $\Theta_Q$ & \tt{float[][]} \\
\end{longtable}

\subsubsection*{Agricultural Agents}
\label{sec:land_farmer_init}
Many of the initial values used by the agricultural agents are initialized
from the input data.
The variables which are set during initialization are listed in
Table~\ref{tab:land_farmer_init}.

\begin{longtable}{lcll}
    \caption{Table listing initialization of parameters of
    the agricultural agents in the land-cover transition model}
    \label{tab:land_farmer_init}
    \\
    \hline\hline
    Value & & Initialization & Type \\
    \hline
    \endfirsthead
    \caption[]{(continued ...)}\\ \hline\hline
    Value & & Initialization & Type\\ \hline
    \endhead
    \hline
    \endfoot
    Parcel Data & $A_x$ & 
    Read from input file & \tt{float} \\
    Corn Production (USD) & $p_c$ 
        & $\Tri_2(3.362551, 4259232)$ & \tt{float} \\
    Corn Production (P) & $p_{p,c}$ 
        & $\Tri_2(\SI{2.02e-4}{}, \SI{6.17e-4}{})$ & \tt{float} \\
    Hay Production (USD) & $p_h$ 
        & $\Tri_2(0.358672, 0.470757)$ & \tt{float} \\
    Hay Production (P) & $p_{p,h}$
        & $\Tri_2(\SI{3.37e-5}{}, \SI{1.12e-4}{})$ & \tt{float} \\
    Beef Production (USD) & $p_b$ & $\Tri_2(900.0, 1200.0)$ & \tt{float} \\
    Dairy Production (USD) & $p_d$ & $\Tri_2(210.0, 250.0)$ & \tt{float} \\
    Cow Production (P) & $p_{p,[bd]}$ 
        & $\Tri_2(\SI{3.366e-4}{}, \SI{7.853e-4}{})$ & \tt{float} \\
    Cows Owned & $C$ & $\Tri_2(3600, 7800)$ & \tt{uint} \\
\end{longtable}

\subsubsection*{Residential and Commercial Agents}

Initial agent salaries are selected from the weighted categorical distribution described in
Table~\ref{tab:sals} on the range $\left(250, 1000\right)$.
The initial rent/mortgage price for each urban land parcel is selected from a uniform distribution
on the range $\left(400, 1100\right)$.

\begin{longtable}{ll}
\caption{Table listing distribution of initial agent salaries and the weight on their probability.}\label{tab:sals}\\
\hline\hline
Weight & Initial Salary \\
\hline\endfirsthead
\hline\hline
Weight & Initial Salary \\
\hline\endhead
\hline\endfoot
    10 & 250 \\
    20 & 300 \\
    30 & 350 \\
    50 & 400 \\
    70 & 450 \\
    85 & 500 \\
    90 & 550 \\
    85 & 600 \\
    70 & 650 \\
    60 & 700 \\
    50 & 750 \\
    50 & 800 \\
    50 & 850 \\
    50 & 900 \\
    30 & 950 \\
    20 & 1000 \\
\end{longtable}
\chapter{Result Listings}
\label{app:results}

This appendix contains listings of model results which would be difficult to display alongside
the main text.

\begin{longtable}{ccccc}
    \caption{Mean BMP~adoption rate for uniform-population runs of the agricultural model for 
    the parameterizations with results plotted in Figure~\ref{fig:farm_res_g00}, Figure~\ref{fig:farm_res_g05}, and
    Figure~\ref{fig:farm_res_g20}.} \\
    \label{tab:full_res_uniform} \\
    \hline
    \hline
    $g$ & $F$ & $BMP_e$ & $\Delta EE$ & Adoption Rate \\
    \hline
    \endfirsthead
    \caption[]{(continued...)}\\
    \hline
    \hline
    $g$ & $F$ & $BMP_e$ & $\Delta EE$ & Adoption Rate \\
    \hline
    \endhead
    \hline
    \endfoot
    \hline
    \endlastfoot
    % g = 0
    % F = 0
    % BMP_e = 0.0
    0.0 & 0.0 & 0.0 & -0.2  & $0.053$ \\
    0.0 & 0.0 & 0.0 &  0.0  & $0.058$ \\
    0.0 & 0.0 & 0.0 &  0.2  & $0.040$ \\ \hline
    % BMP_e = 0.5
    0.0 & 0.0 & 0.5 & -0.2  & $0.150$ \\
    0.0 & 0.0 & 0.5 &  0.0  & $0.252$ \\
    0.0 & 0.0 & 0.5 &  0.2  & $0.461$ \\ \hline
    % BMP_e = 1.0
    0.0 & 0.0 & 1.0 & -0.2  & $0.349$ \\
    0.0 & 0.0 & 1.0 &  0.0  & $0.505$ \\
    0.0 & 0.0 & 1.0 &  0.2  & $0.759$ \\ \hline
    % F = 0.5
    % BMP_e = 0.0
    0.0 & 0.5 & 0.0 & -0.2  & $0.173$ \\
    0.0 & 0.5 & 0.0 &  0.0  & $0.231$ \\
    0.0 & 0.5 & 0.0 &  0.2  & $0.452$ \\ \hline
    % BMP_e = 0.5
    0.0 & 0.5 & 0.5 & -0.2  & $0.202$ \\
    0.0 & 0.5 & 0.5 &  0.0  & $0.300$ \\
    0.0 & 0.5 & 0.5 &  0.2  & $0.450$ \\ \hline
    % BMP_e = 1.0
    0.0 & 0.5 & 1.0 & -0.2  & $0.399$ \\
    0.0 & 0.5 & 1.0 &  0.0  & $0.488$ \\
    0.0 & 0.5 & 1.0 &  0.2  & $0.589$ \\ \hline
    % F = 1.0
    % BMP_e = 0.0
    0.0 & 1.0 & 0.0 & -0.2  & $0.310$ \\
    0.0 & 1.0 & 0.0 &  0.0  & $0.333$ \\
    0.0 & 1.0 & 0.0 &  0.2  & $0.427$ \\ \hline
    % BMP_e = 0.5
    0.0 & 1.0 & 0.5 & -0.2  & $0.339$ \\
    0.0 & 1.0 & 0.5 &  0.0  & $0.349$ \\
    0.0 & 1.0 & 0.5 &  0.2  & $0.450$ \\ \hline
    % BMP_e = 1.0
    0.0 & 1.0 & 1.0 & -0.2  & $0.411$ \\
    0.0 & 1.0 & 1.0 &  0.0  & $0.491$ \\
    0.0 & 1.0 & 1.0 &  0.2  & $0.529$ \\ \hline
    % g = 0.05
    % F = 0
    % BMP_e = 0.0
    0.05 & 0.0 & 0.0 & -0.2  & $0.106$ \\
    0.05 & 0.0 & 0.0 &  0.0  & $0.124$ \\
    0.05 & 0.0 & 0.0 &  0.2  & $0.121$ \\ \hline
    % BMP_e = 0.5
    0.05 & 0.0 & 0.5 & -0.2  & $0.364$ \\
    0.05 & 0.0 & 0.5 &  0.0  & $0.418$ \\
    0.05 & 0.0 & 0.5 &  0.2  & $0.478$ \\ \hline
    % BMP_e = 1.0
    0.05 & 0.0 & 1.0 & -0.2  & $0.733$ \\
    0.05 & 0.0 & 1.0 &  0.0  & $0.793$ \\
    0.05 & 0.0 & 1.0 &  0.2  & $0.847$ \\ \hline
    % F = 0.5
    % BMP_e = 0.0
    0.05 & 0.5 & 0.0 & -0.2  & $0.092$ \\
    0.05 & 0.5 & 0.0 &  0.0  & $0.088$ \\
    0.05 & 0.5 & 0.0 &  0.2  & $0.094$ \\ \hline
    % BMP_e = 0.5
    0.05 & 0.5 & 0.5 & -0.2  & $0.219$ \\
    0.05 & 0.5 & 0.5 &  0.0  & $0.265$ \\
    0.05 & 0.5 & 0.5 &  0.2  & $0.332$ \\ \hline
    % BMP_e = 1.0
    0.05 & 0.5 & 1.0 & -0.2  & $0.378$ \\
    0.05 & 0.5 & 1.0 &  0.0  & $0.463$ \\
    0.05 & 0.5 & 1.0 &  0.2  & $0.566$ \\ \hline
%    % F = 1.0
%    % BMP_e = 0.0
%    0.05 & 1.0 & 0.0 & -0.2  & $0.$ \\
%    0.05 & 1.0 & 0.0 &  0.0  & $0.$ \\
%    0.05 & 1.0 & 0.0 &  0.2  & $0.$ \\ \hline
%    % BMP_e = 0.5
%    0.05 & 1.0 & 0.5 & -0.2  & $0.$ \\
%    0.05 & 1.0 & 0.5 &  0.0  & $0.$ \\
%    0.05 & 1.0 & 0.5 &  0.2  & $0.$ \\ \hline
%    % BMP_e = 1.0
%    0.05 & 1.0 & 1.0 & -0.2  & $0.$ \\
%    0.05 & 1.0 & 1.0 &  0.0  & $0.$ \\
%    0.05 & 1.0 & 1.0 &  0.2  & $0.$ \\ \hline
    % g = 0.2
    % F = 0
    % BMP_e = 0.0
    0.2 & 0.0 & 0.0 & -0.2  & $0.304$ \\
    0.2 & 0.0 & 0.0 &  0.0  & $0.276$ \\
    0.2 & 0.0 & 0.0 &  0.2  & $0.361$ \\ \hline
    % BMP_e = 0.5
    0.2 & 0.0 & 0.5 & -0.2  & $0.396$ \\
    0.2 & 0.0 & 0.5 &  0.0  & $0.473$ \\
    0.2 & 0.0 & 0.5 &  0.2  & $0.548$ \\ \hline
    % BMP_e = 1.0
    0.2 & 0.0 & 1.0 & -0.2  & $0.667$ \\
    0.2 & 0.0 & 1.0 &  0.0  & $0.695$ \\
    0.2 & 0.0 & 1.0 &  0.2  & $0.733$ \\ \hline
    % F = 0.5
    % BMP_e = 0.0
    0.2 & 0.5 & 0.0 & -0.2  & $0.333$ \\
    0.2 & 0.5 & 0.0 &  0.0  & $0.356$ \\
    0.2 & 0.5 & 0.0 &  0.2  & $0.454$ \\ \hline
    % BMP_e = 0.5
    0.2 & 0.5 & 0.5 & -0.2  & $0.396$ \\
    0.2 & 0.5 & 0.5 &  0.0  & $0.467$ \\
    0.2 & 0.5 & 0.5 &  0.2  & $0.574$ \\ \hline
    % BMP_e = 1.0
    0.2 & 0.5 & 1.0 & -0.2  & $0.5957$ \\
    0.2 & 0.5 & 1.0 &  0.0  & $0.586$ \\
    0.2 & 0.5 & 1.0 &  0.2  & $0.641$ \\ \hline
    % F = 1.0
    % BMP_e = 0.0
    0.2 & 1.0 & 0.0 & -0.2  & $0.324$ \\
    0.2 & 1.0 & 0.0 &  0.0  & $0.340$ \\
    0.2 & 1.0 & 0.0 &  0.2  & $0.405$ \\ \hline
    % BMP_e = 0.5
    0.2 & 1.0 & 0.5 & -0.2  & $0.396$ \\
    0.2 & 1.0 & 0.5 &  0.0  & $0.478$ \\
    0.2 & 1.0 & 0.5 &  0.2  & $0.590$ \\ \hline
    % BMP_e = 1.0
    0.2 & 1.0 & 1.0 & -0.2  & $0.520$ \\
    0.2 & 1.0 & 1.0 &  0.0  & $0.608$ \\
    0.2 & 1.0 & 1.0 &  0.2  & $0.661$ \\
\end{longtable}

Figure~\ref{fig:farm_res_mix0}
\begin{longtable}{ccccccc}
    \caption{Mean BMP~adoption rate for mixed-population runs of the agricultural model for the
    parameterizations with results plotted in Figure~\ref{fig:farm_res_mix0} for each sub-population:
    (1)~local~$F = 0$, (2)~local~$F = 1$, and (3)~mixed neighborhood.}
%    \caption[Results of Plotted Mix Population Runs]{Results of Plotted Mixed Population Runs} \\
    \label{tab:plot_res_mix} \\
    \hline
    \hline
    $g$ & Group & $P$ & $BMP_e$ & $\Delta EE$ & Adoption Rate \\
    \hline
    \endfirsthead
    \caption[]{(continued)}\\
    \hline
    \endhead
    \hline
    \endfoot
    0.0 & 1 & 0.25 & 0.0 &  0.0  & $0.111$ \\
    0.0 & 2 & 0.25 & 0.0 &  0.0  & $0.202$ \\
    0.0 & 3 & 0.25 & 0.0 &  0.0  & $0.209$ \\ \hline

    0.0 & 1 & 0.25 & 0.5 &  0.0  & $0.270$ \\
    0.0 & 2 & 0.25 & 0.5 &  0.0  & $0.322$ \\
    0.0 & 3 & 0.25 & 0.5 &  0.0  & $0.313$ \\ \hline

    0.0 & 1 & 0.25 & 1.0 &  0.0  & $0.507$ \\
    0.0 & 2 & 0.25 & 1.0 &  0.0  & $0.529$ \\
    0.0 & 3 & 0.25 & 1.0 &  0.0  & $0.508$ \\ \hline

    0.0 & 1 & 0.5 & 0.0 &  0.0  & $0.140$ \\
    0.0 & 2 & 0.5 & 0.0 &  0.0  & $0.186$ \\
    0.0 & 3 & 0.5 & 0.0 &  0.0  & $0.166$ \\ \hline

    0.0 & 1 & 0.5 & 0.5 &  0.0  & $0.265$ \\
    0.0 & 2 & 0.5 & 0.5 &  0.0  & $0.302$ \\
    0.0 & 3 & 0.5 & 0.5 &  0.0  & $0.289$ \\ \hline

    0.0 & 1 & 0.5 & 1.0 &  0.0  & $0.504$ \\
    0.0 & 2 & 0.5 & 1.0 &  0.0  & $0.537$ \\
    0.0 & 3 & 0.5 & 1.0 &  0.0  & $0.522$ \\ \hline

    0.0 & 1 & 0.75 & 0.0 &  0.0  & $0.111$ \\
    0.0 & 2 & 0.75 & 0.0 &  0.0  & $0.199$ \\
    0.0 & 3 & 0.75 & 0.0 &  0.0  & $0.149$ \\ \hline

    0.0 & 1 & 0.75 & 0.5 &  0.0  & $0.271$ \\
    0.0 & 2 & 0.75 & 0.5 &  0.0  & $0.323$ \\
    0.0 & 3 & 0.75 & 0.5 &  0.0  & $0.321$ \\ \hline

    0.0 & 1 & 0.75 & 1.0 &  0.0  & $0.518$ \\
    0.0 & 2 & 0.75 & 1.0 &  0.0  & $0.546$ \\
    0.0 & 3 & 0.75 & 1.0 &  0.0  & $0.518$ \\ \hline
\end{longtable}

%\begin{longtable}{ccccccc}
%    \caption[Results from Mixed Runs]{Results from Mixed Runs} \\
%    \label{tab:full_res_mixed} \\
%    \hline
%    \hline
%    $P$ & Group & $g$ & $BMP_e$ & $\Delta EE$ & Adoption Rate \\
%    \hline
%    \endfirsthead
%    \caption[]{(continued)}\\
%    \hline
%    \hline
%    \endhead
%    \hline
%    \endfoot
%    \hline
%    \endlastfoot
%    % P = 0.25
%    % g = 0.0
%    % BMP_e = 0.0
%    % DEE = -0.2
%    0.25 & A & 0.0 & 0.0 & -0.2 & $0.000\pm 0.000$ \\
%    0.25 & B & 0.0 & 0.0 & -0.2 & $0.000\pm 0.000$ \\
%    0.25 & C & 0.0 & 0.0 & -0.2 & $0.000\pm 0.000$ \\
%    % DEE = -0.15
%    0.25 & A & 0.0 & 0.0 & -0.15 & $0.000\pm 0.000$ \\
%    0.25 & B & 0.0 & 0.0 & -0.15 & $0.000\pm 0.000$ \\
%    0.25 & C & 0.0 & 0.0 & -0.15 & $0.000\pm 0.000$ \\
%    % DEE = -0.1
%    0.25 & A & 0.0 & 0.0 & -0.1 & $0.000\pm 0.000$ \\
%    0.25 & B & 0.0 & 0.0 & -0.1 & $0.000\pm 0.000$ \\
%    0.25 & C & 0.0 & 0.0 & -0.1 & $0.000\pm 0.000$ \\
%    % DEE = -0.05
%    0.25 & A & 0.0 & 0.0 & -0.05 & $0.000\pm 0.000$ \\
%    0.25 & B & 0.0 & 0.0 & -0.05 & $0.000\pm 0.000$ \\
%    0.25 & C & 0.0 & 0.0 & -0.05 & $0.000\pm 0.000$ \\
%    % DEE = 0.0
%    0.25 & A & 0.0 & 0.0 & 0.0 & $0.000\pm 0.000$ \\
%    0.25 & B & 0.0 & 0.0 & 0.0 & $0.000\pm 0.000$ \\
%    0.25 & C & 0.0 & 0.0 & 0.0 & $0.000\pm 0.000$ \\
%    % DEE = 0.05
%    0.25 & A & 0.0 & 0.0 & 0.05 & $0.000\pm 0.000$ \\
%    0.25 & B & 0.0 & 0.0 & 0.05 & $0.000\pm 0.000$ \\
%    0.25 & C & 0.0 & 0.0 & 0.05 & $0.000\pm 0.000$ \\
%    % DEE = 0.1
%    0.25 & A & 0.0 & 0.0 & 0.1 & $0.000\pm 0.000$ \\
%    0.25 & B & 0.0 & 0.0 & 0.1 & $0.000\pm 0.000$ \\
%    0.25 & C & 0.0 & 0.0 & 0.1 & $0.000\pm 0.000$ \\
%    % DEE = 0.15
%    0.25 & A & 0.0 & 0.0 & 0.15 & $0.000\pm 0.000$ \\
%    0.25 & B & 0.0 & 0.0 & 0.15 & $0.000\pm 0.000$ \\
%    0.25 & C & 0.0 & 0.0 & 0.15 & $0.000\pm 0.000$ \\
%    % DEE = 0.2
%    0.25 & A & 0.0 & 0.0 & 0.2 & $0.000\pm 0.000$ \\
%    0.25 & B & 0.0 & 0.0 & 0.2 & $0.000\pm 0.000$ \\
%    0.25 & C & 0.0 & 0.0 & 0.2 & $0.000\pm 0.000$ \\
%    % P = 0.5
%    % P = 0.75
%\end{longtable}
%
%%\begin{table}[h]
%%\centering
%%\begin{longtable}{lclllll}
%%    \caption[Full Results... 1]{Full Table} \\
%%\begin{tabularx}{\textwidth}{lcllllX}
%%\hline
%%$F$ & BMP efficacy & $E$ & Mean & Std. Dev. & Confidence & Never \\
%%\hline
%%\endfirsthead
%%\hline
%%$F$ & BMP efficacy & $E$ & Mean & Std. Dev. & Confidence & Never \\
%%\hline
%%\endhead
%%\hline
%%\endfoot
%%\hline
%%\endlastfoot
%%% F = 0
%%% Efficacy = 0.0
%%0.0 & 0.0 & -0.2 & 0.000 & 0.000 & 0.000 & 0 \\
%%0.0 & 0.0 & -0.15 & 0.000 & 0.000 & 0.000 & 0 \\
%%0.0 & 0.0 & -0.1 & 0.000 & 0.000 & 0.000 & 0 \\
%%0.0 & 0.0 & -0.05 & 0.000 & 0.000 & 0.000 & 0 \\
%%0.0 & 0.0 &  0.0 & 0.000 & 0.000 & 0.000 & 0 \\
%%0.0 & 0.0 &  0.05 & 0.000 & 0.000 & 0.000 & 0 \\
%%0.0 & 0.0 &  0.1 & 0.000 & 0.000 & 0.000 & 0 \\
%%0.0 & 0.0 &  0.15 & 0.000 & 0.000 & 0.000 & 0 \\
%%0.0 & 0.0 &  0.2 & 0.000 & 0.000 & 0.000 & 0 \\
%%
%%% Efficacy = 0.1
%%\hline
%%0.0 & 0.1 & -0.2 & 0.000 & 0.000 & 0.000 & 0 \\
%%0.0 & 0.1 & -0.15 & 0.000 & 0.000 & 0.000 & 0 \\
%%0.0 & 0.1 & -0.1 & 0.000 & 0.000 & 0.000 & 0 \\
%%0.0 & 0.1 & -0.05 & 0.000 & 0.000 & 0.000 & 0 \\
%%0.0 & 0.1 &  0.0 & 0.000 & 0.000 & 0.000 & 0 \\
%%0.0 & 0.1 &  0.05 & 0.000 & 0.000 & 0.000 & 0 \\
%%0.0 & 0.1 &  0.1 & 0.000 & 0.000 & 0.000 & 0 \\
%%0.0 & 0.1 &  0.15 & 0.000 & 0.000 & 0.000 & 0 \\
%%0.0 & 0.1 &  0.2 & 0.000 & 0.000 & 0.000 & 0 \\
%%
%%% F = 0
%%% Efficacy = 0.2
%%0.0 & 0.2 & -0.2 & 0.000 & 0.000 & 0.000 & 0 \\
%%0.0 & 0.2 & -0.15 & 0.000 & 0.000 & 0.000 & 0 \\
%%0.0 & 0.2 & -0.1 & 0.000 & 0.000 & 0.000 & 0 \\
%%0.0 & 0.2 & -0.05 & 0.000 & 0.000 & 0.000 & 0 \\
%%0.0 & 0.2 &  0.0 & 0.000 & 0.000 & 0.000 & 0 \\
%%0.0 & 0.2 &  0.05 & 0.000 & 0.000 & 0.000 & 0 \\
%%0.0 & 0.2 &  0.1 & 0.000 & 0.000 & 0.000 & 0 \\
%%0.0 & 0.2 &  0.15 & 0.000 & 0.000 & 0.000 & 0 \\
%%0.0 & 0.2 &  0.2 & 0.000 & 0.000 & 0.000 & 0 \\
%%
%%% Efficacy = 0.3
%%\hline
%%0.0 & 0.3 & -0.2 & 0.000 & 0.000 & 0.000 & 0 \\
%%0.0 & 0.3 & -0.15 & 0.000 & 0.000 & 0.000 & 0 \\
%%0.0 & 0.3 & -0.1 & 0.000 & 0.000 & 0.000 & 0 \\
%%0.0 & 0.3 & -0.05 & 0.000 & 0.000 & 0.000 & 0 \\
%%0.0 & 0.3 &  0.0 & 0.000 & 0.000 & 0.000 & 0 \\
%%0.0 & 0.3 &  0.05 & 0.000 & 0.000 & 0.000 & 0 \\
%%0.0 & 0.3 &  0.1 & 0.000 & 0.000 & 0.000 & 0 \\
%%0.0 & 0.3 &  0.15 & 0.000 & 0.000 & 0.000 & 0 \\
%%0.0 & 0.3 &  0.2 & 0.000 & 0.000 & 0.000 & 0 \\
%%
%%% Efficacy = 0.4
%%\hline
%%0.0 & 0.4 & -0.2 & 0.000 & 0.000 & 0.000 & 0 \\
%%0.0 & 0.4 & -0.15 & 0.000 & 0.000 & 0.000 & 0 \\
%%0.0 & 0.4 & -0.1 & 0.000 & 0.000 & 0.000 & 0 \\
%%0.0 & 0.4 & -0.05 & 0.000 & 0.000 & 0.000 & 0 \\
%%0.0 & 0.4 &  0.0 & 0.000 & 0.000 & 0.000 & 0 \\
%%0.0 & 0.4 &  0.05 & 0.000 & 0.000 & 0.000 & 0 \\
%%0.0 & 0.4 &  0.1 & 0.000 & 0.000 & 0.000 & 0 \\
%%0.0 & 0.4 &  0.15 & 0.000 & 0.000 & 0.000 & 0 \\
%%0.0 & 0.4 &  0.2 & 0.000 & 0.000 & 0.000 & 0 \\
%%
%%% Efficacy = 0.5
%%\hline
%%0.0 & 0.5 & -0.2 & 0.000 & 0.000 & 0.000 & 0 \\
%%0.0 & 0.5 & -0.15 & 0.000 & 0.000 & 0.000 & 0 \\
%%0.0 & 0.5 & -0.1 & 0.000 & 0.000 & 0.000 & 0 \\
%%0.0 & 0.5 & -0.05 & 0.000 & 0.000 & 0.000 & 0 \\
%%0.0 & 0.5 &  0.0 & 0.000 & 0.000 & 0.000 & 0 \\
%%0.0 & 0.5 &  0.05 & 0.000 & 0.000 & 0.000 & 0 \\
%%0.0 & 0.5 &  0.1 & 0.000 & 0.000 & 0.000 & 0 \\
%%0.0 & 0.5 &  0.15 & 0.000 & 0.000 & 0.000 & 0 \\
%%0.0 & 0.5 &  0.2 & 0.000 & 0.000 & 0.000 & 0 \\
%%
%%\end{longtable}
%%%\end{tabularx}
%%%\end{table}


\end{document} 
